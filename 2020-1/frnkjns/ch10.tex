\section*{Chapter 10.}

Problem 7. \\

Let ${1\over r} = {\theta \over p} + {1 - \theta \over q}$. Then

\begin{equation*}
	\begin{split}
		\int |f|^r d\mu & = \int |f|^{r\theta + \left( 1-\theta \right)r} d\mu \\
		& \leq \left( \int |f|^p d\mu \right)^{\frac{r\theta}{p}} \left( \int |g|^q d\mu \right) ^{\frac{\left( 1-\theta \right)r}{q}}
	\end{split}
	\label{<+label+>}
\end{equation*}

So, $\| f \| _r \leq \|f \|_p^{\theta} \| f \|_q^{1-\theta}$.\\

Problem 9. \\

Consider $k^{1 \over p_0} 1_{\left( 0, {1\over k} \right)}$. \\

Problem 10. \\

Using previous lemma. Choose $f_{n_{k}}$ : subsequence which converges to $g$ $\mu$-a.e. That subsequence converges to $h$ in $L_{p_2}$. So, we can choose $f_{n_{k_i}}$ : subsequnce of subsequence which converges to $h$ $\mu$-a.e. Therefore $g$ and $h$ are same $\mu$-a.e. \\

Problem 11. \\

We already know that $|f| \leq \|f \|_{\infty}$ a.e. So let $N$ be corresponding null set. Then $|f 1_{N^c}| \leq \| f \| _{\infty}$ for all $x \in X$. Also $f 1_{N^c} = f$ a.e.

If $f=g$ a.e, $|f| \leq \sup |g|$ a.e. Then $\| f \| _{\infty} \leq \sup |g|$. \\

Problem 12. \\

For continuous $f$, $|f| \leq M$ a.e. implies $|f| \leq M$ for all $x\in X$. By using this, done. \\

Problem 13. \\

\begin{equation*}
	\int |f|^r d\mu = \int |f|^{r-p} |f|^p d\mu \leq \| f \|_{\infty}^{r-p} \| f \|_p ^p
	\label{<+label+>}
\end{equation*}

Therefore $\| f \| _r \leq \| f \|_{\infty}^{1- {p\over r}} \| f \|_p ^{p\over r}$.
So $L_p \cap L_{\infty} \subset L_r$. \\

Problem 16. \\

$f=1$ is in $L_{\infty}$ but not in all of $L_p$ for $1\leq p < \infty$. 

\begin{equation*}
	\int |f|^p d\mu = \int |f|^p 1 d\mu \leq \|f \|_{\infty}^p \mu\left( X \right)
	\label{<+label+>}
\end{equation*}

So $\|f \|_p \leq \| f \|_{\infty} \mu\left( X \right)^{1\over p}$. By taking $\limsup$ both sides, we get $\limsup \| f \|_p \leq \| f \|_{\infty}$.

On the other hand, let $\| f \|_{\infty} > t > 0$. Then by definition, $\mu\left( |f| > t  \right) >0$. So we get 
\begin{equation*}
	\int |f|^p d\mu \geq \int_{|f| > t} |f|^p d\mu \geq t^p \mu\left( |f| > t \right)
	\label{<+label+>}
\end{equation*}

So $\|f \|_p \geq t \mu\left( |f| > t \right)^{1\over p}$. By taking $\liminf$ both sides, we get $\liminf \| f \|_p \geq t$. But $t$ is arbitrary, $\liminf \|f\|_p \geq \|f\|_{\infty}$. \\

Problem 20. \\

Let $E_x = \left[ x, x+1 \right]$. Then 

\begin{equation*}
	\begin{split}
		\int |f1_{E_x} | d\mu & = \int |f1_{E_x}1_{E_x}| d\mu \\
		& \leq \left ( \int |f 1_{E_x}|^p d\mu \right )^{1\over p} \mu\left( E_x \right)^{1\over q} \\
		& \leq \| f \|_p < \infty
	\end{split}
	\label{<+label+>}
\end{equation*}

But note that $|f 1_{E_x} |^p \leq |f|^p \in L_1$. Then by DCT, $\lim_{x\rightarrow \infty} \left | \int f 1_{E_x} d\mu \right | \leq \lim_{x\rightarrow \infty}\int | f 1_{E_x} |^p d\mu = 0$.\\

Problem 21. \\

Consider $\sqrt{f}$ and ${1\over \sqrt{f}}$. \\

Problem 22. \\

Use Fatou's Lemma. \\

Problem 23. \\

Convexity of $|x|^p$ implies results of (a).

Using generalization of DCT, we can get (b). \\

Problem 24. \\

Assume $f_k \rightarrow f$ in $L_p$. Then $\lim_{k\rightarrow \infty} \| f-f_k \|_p = 0$. But $| \|f_k \|_p - \|f \|_p | \leq \|f_k - f \|_p$. 

On the other hand, assume $\lim_{k\rightarrow \infty}\|f_k\|_p = \|f\|_p$. Using the result of problem 23, done. \\

Problem 30. \\

If $p < \infty$, Let $\|f \|_{\infty} > t > 0$. Then $\int |f|^p d\mu \geq t^q \mu\left( |f| > t \right) \geq t^q$. Therefore $\| f \|_p \geq \|f \|_{\infty}$. 

Let $p < q < \infty$. Then $\int |f|^q d\mu = \int |f|^{q-p} |f|^p d\mu \leq \|f\|_{\infty}^{q-p} \|f\|_{p}^p$. Since $p < \infty$, $\| f \| _{\infty} \leq \|f \|_p$. So, done. \\

Problem 36. \\

Without loss of generality, we can assume $\| f+g \|_p =1$. Then
\begin{equation*}
	\|f+g\|_p^p = \int (f+g)^{p-1}(f+g)d\mu \geq \left( \int (f+g)^{p'(p-1)}d\mu \right)^{1\over p'}\left( \|f\|_p + \|g\|_p \right)
	\label{<+label+>}
\end{equation*}

So result follows. \\

Problem 37. \\

Let $A = \left\{ |f| \geq 1 \right\}$ and $B = \left\{ 0 < |f| < 1 \right\}$ and $C = \left\{ f = 0 \right\}$, $\varphi\left( E \right) = \int_E |f|^p d\mu$. Then $\varphi\left( X \right) = \varphi\left( A \right) + \varphi\left( B \right) + \varphi\left( C \right)$. 

For $A$, $|f|^p \leq |f|^{p_0} \in L_1$. So $\lim_{p \rightarrow 0} \varphi\left( A \right) = \mu\left( A \right)$ by DCT because $|f|^p \rightarrow 1$ as $p \rightarrow 0$.

For $B$, as $p \rightarrow 0$, $|f|^p \uparrow 1$. By MCT, $\lim_{p \rightarrow 0} \int_B |f|^p d\mu = \mu\left( B \right)$. 

For $C$, $\varphi\left( C \right) = 0$. Therefore

\begin{equation*}
	\lim_{p\rightarrow 0}\int_X |f|^p d\mu = \mu\left( A \right) + \mu\left( B \right) = \mu\left( \left\{ |f| \ne 0 \right\} \right)
\end{equation*}
\hfill \\

Problem 38. \\

Since $e^x$ is continuous, it is enough to show that $\lim_{p\rightarrow 0} \log \| f \|_p = \int_X \log |f| d\mu$. From concavity of $\log x$, ${1\over p} \log x$ is also concave function. So, ${1\over p}\log \int |f|^p d\mu \geq \int_X {1\over p}\log |f|^p d\mu$ for each $p$. Therefore
\begin{equation*}
	\lim_{p \rightarrow 0} \log \|f \|_p \geq \int_X \log |f| d\mu
	\label{<+label+>}
\end{equation*}

Note that $\log x \leq x-1$ for all $x > 0$. By using this inequality, $\log \| f \|_p = {1\over p}\log \| f \|_p ^p \leq {1\over p}\left( \| f \|_p ^p - 1 \right) = {1\over p}\int\left( |f|^p -1 \right)d\mu$.

Let $A = \left\{ |f| \geq 1 \right\}$ and $B = \left\{ 0 < |f| < 1 \right\}$. For $A$, $\frac{|f|^p - 1}{p} \leq \frac{|f|^{p_0} - 1}{p_0}\in L_1$ so we can use DCT. For $B$, $\frac{1-|f|^p}{p} \uparrow -\log|f|$ so we can use MCT. Therefore
\begin{equation*}
	\begin{split}
		\lim_{p\rightarrow 0} \log \| f \|_p & \leq \lim_{p \rightarrow 0} \int_A \frac{|f|^p - 1}{p}d\mu - \lim_{p\rightarrow 0}\int_B \frac{1-|f|^p}{p}d\mu \\
		& = \int_A \log |f| d\mu + \int_B \log |f| d\mu \\
		& = \int_X \log |f| d\mu.
	\end{split}
	\label{<+label+>}
\end{equation*}

If $\mu\left( |f| = 0 \right) > 0$, we can interpret this situation as $\lim_{p\rightarrow 0} \| f\|_p = 0$.\\ 

When $p_0 = \infty$, slight modification of above is needed. (For problem 37, 38)
