\section*{chapter 2: Lebesgue Measure on $\mathbb{R}^n$}
\subsection*{section A: Construction}\hfill \\

In this section, we will construct Lebesgue measure from set of special rectangles. Although there are many other methods to consturct measure on Euclidean space, method in our textbook looks like intuitive and good to study for beginners. Other textbooks use Caratéodory extension or Urysohn lemma to construct Lebesgue measure on Euclidean space. \\

Problem 3. \\

Consider the open set $G \setminus P$ and $x$ in that set. There is $\varepsilon$ neighborhood of $x$ contained in $G \setminus P$. But $\varepsilon$ neighborhood contains special rectangle $I$  whose measure is positive. Now, take $P' = P \cup I$.\\

Problem 5. 'at most countably many disjoint open sets'\\

Let $G_i$ be nonempty set. Pick $x \in G_i$ and consider $\varepsilon$ neighborhood contained in $G_i$. That neighborhood contains point $q_i$  whose components are all rational. Consider the injection $G_i \mapsto q_i$ from $\mathcal{I}$ to countable set. (It is clearly injection because each $G_i$ is disjoint.)\\

Problem 6. 'the structure of open sets in real line' \\

Consider an equivalence relation $x \sim y \leftrightarrow x, y \text{ belongs to some open interval contained in }G$. Equivalence class of $x$ is a largest open interval containing $x$.\\

Problem 8. 'open disk cannot be expressed as disjoint union of open rectangles'\\

It is trivial since open disk is connected.\\

Note that regular space with countable basis, metrizable space, compact Hausdorff space are normal space. So Euclidean space is normal space. This fact is useful to prove property 4 of Lebesgue measure on compact sets.\\

Problem 17. \\

Let $x = \sum_{k=1} ^ \infty {\alpha_k \over 3^k} $ where $\alpha_j \in \{ 0, 2\} $. Then $1-x = \sum_{k=1}^\infty {2 - \alpha_k \over 3^k } $ and $2-\alpha_k \in \{ 0, 2 \}$. Therefore, $1-x \in C \leftrightarrow x\in C$.\\



