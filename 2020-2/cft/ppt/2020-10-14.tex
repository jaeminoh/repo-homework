\documentclass{beamer}

\usepackage[utf8]{inputenc}

\title{Conformal Self Mappings}

\author{Jaemin Oh}
\institute{KAIST}
\date{\today}

\begin{document}

\frame{\titlepage}

\begin{frame}
	\frametitle{Contents}
	\tableofcontents
\end{frame}

\begin{frame}
	\frametitle{conformal map}
	\begin{itemize}
		\item $U, V$ : open subsets of $\mathbb{C}$
		\item $f$ is a function of $U$ into $V$
		\item $f$ is conformal if $f$ is bijective and holomorphic.
		\item conformal = biholomorphic
		\item If $h$ is holomorphic function of $U$, and $U$ is somewhat complicated, by considering $h \circ f$, we can change the domain of $h$.
		\item why conformal? it preserves the angle.
	\end{itemize}
\end{frame}

\begin{frame}
	\frametitle{characterizing conformal self mapping of $\mathbb{C}$}
	\begin{itemize}
		\item natural example: $az + b$ for $a \ne 0$
		\item In fact, above form is all of them.
		\item Note that we are considering not just entire function. Conformal self mappings of $\mathbb{C}$ has more condition than entire function.
			\begin{alertblock}{Lemma 6.1.2.}
				If $f: \mathbb{C} \rightarrow \mathbb{C}$ is a conformal then $\lim_{|z| \rightarrow \infty} |f(z)| = \infty$.
\end{alertblock}
	\end{itemize}
\end{frame}

\begin{frame}
	\frametitle{proof of lemma 6.1.2.}
	\begin{itemize}
		\item Fix $M$. We want to show existence of $N$ such that $f\left( \left\{ z: |z| > N \right\} \right) \subset \left\{ w: |w| > M \right\}$. 

		\item Above is equivalent to $\left\{ z : |z| > N \right\} \subset f^{-1}\left( \left\{ w: |w|>N \right\} \right)$ since $f$ is bijective. 

		\item Above is equivalent to $\left\{ z : |z| \leq N \right\} \supset f^{-1}\left( \left\{ w: |w| \leq M \right\} \right)$ by taking complement. 

		\item Existence of $N$ is clear since RHS of above is compact($\Rightarrow$ closed and bounded). 

	\end{itemize}
\end{frame}

\begin{frame}
	\frametitle{characterizing conformal self mapping of $\mathbb{C}$}


	 We already know that $f$ must be polynomial when $f$ is entire and $f \rightarrow \infty$ as $|z| \rightarrow \infty$. Using this and fundamental thm of algebra, we can characterize conformal self mapping of complex plane. \\
	\pause

	\begin{alertblock}{lemma 6.1.3}
		$f$ is a conformal self mapping of $\mathbb{C}$. Then there are $B, D >0$ such that $|z| > D$ implies $|f(z)| < B|z|$.
		
	\end{alertblock}

\end{frame}

\begin{frame}
	\frametitle{proof of lemma 6.1.3.}
	\begin{itemize}
		\item There is $C$ such that $|z|>C$ implies $|f(z)| >1$.
		\item Define $g(z) = 1/f\left( {1 \over z} \right)$ for $z \in D'(0, \frac{1}{C})$.
		\item As $z \rightarrow 0$, $g \rightarrow 0$. So $g$ has removable singularity at $0$.
		\item $g'(0) \ne 0$ since $g$ is injection since $f$ is injection.
		\item Near $0$, $\left | \frac{g(z)}{z} \right | > \frac{1}{B}$. 
		\item Near $0$, $\left | f(\frac{1}{z}) \right | < \frac{B}{|z|}$.
		\item $z \mapsto \frac{1}{z}$ leads the conclusion.
	\end{itemize}

\end{frame}

\begin{frame}
	\frametitle{characterizing}
	\begin{itemize}
		\item Consider $|f^{(n)}(0)| \leq \frac{n!}{r^n}\sup_{w \in \partial D(0, r)}|f(w)|$.
		\item For $r > D$, supremum above $\leq Br$ by lemma 6.1.3.
		\item If $n >1$, by letting $r \rightarrow \infty$, $n$-th derivative of $f$ at $0$ must be zero.
		\item Therefore $f$ must be polynomial of degree at most $1$.
		\item But $f$ must be nonconstant. So $f(z) = az + b $ for $a \ne 0$.
		\item We are characterized conformal self mappings of $\mathbb{C}$.
	\end{itemize}
\end{frame}

\begin{frame}
	\frametitle{remark}
	\begin{itemize}
		\item $h$ is holomorphic on $\left\{ z: |z| > \alpha \right\}$ and $\lim_{|z| \rightarrow \infty} |h(z)| = \infty$.
		\item By same procedure in proof of lemma 6.1.3, we can conclude that there are $B, D>0$ such that $|z| > D \Rightarrow |h(z)| < B|z|^n$ for some $n$.
		\item Why $n$? Because we cannot say $g'(0) \ne 0$. But, $g^{(n)}(0) \ne 0$ for some $n$ since $g$ is nonconstant since $h$ is nonconstant. *$g(z) = 1/h(1/z)$
		\item Note that entire function $\varphi$ which satisfies $\lim_{|z| \rightarrow \infty} |\varphi(z) | = \infty$ must be polynomial.
	\end{itemize}
\end{frame}

\begin{frame}
	\frametitle{characterizing conformal self mapping of unit disc}
	\begin{itemize}
		\item natural example : rotation ($f(z) = wz$ for $|w| = 1$)
		\item In fact, above form is all of them which fixes origin.
		\item 
			\begin{alertblock}{lemma 6.2.1.}
				$f: D \rightarrow D$ is biholomorphic which fixes origin iff $f(z) = wz$ for $|w| = 1$
				
			\end{alertblock}
	\end{itemize}
\end{frame}

\begin{frame}
	\frametitle{proof of lemma 6.2.1.}
	\begin{itemize}
		\item $g = f^{-1}$. Then both of $f, g$ are fixing origin.
		\item Schwarz lemma says $|f'(0)| $ and $|g'(0)|$ are $\leq 1$.
		\item Chain rule says $f'(0)g'(0) = 1$. This leads $|f'(0)| = |g'(0)| = 1$. 
		\item Uniqueness of Schwarz lemma tells us that $f(z) = f'(0)z$.
	\end{itemize}
\end{frame}

\begin{frame}
	\frametitle{Mobius transformation}
	\begin{itemize}
		\item $\varphi_a(z) = \frac{z-a}{1-\bar{a}z}$ for $|a| <1$ is called Mobius transformation. 

		\item Theorem 5.5.2 says $\varphi_a$ is conformal self mapping of unit disc. So we take it for granted.

		\item 
			\begin{alertblock}{theorem 6.2.3.}
				$f$ is conformal self mapping of unit disc. Then $f(z) = w \varphi_a(z)$ for some $|a|<1$ and $|w| = 1$. 
				
			\end{alertblock}
	\end{itemize}

\end{frame}

\begin{frame}
	\frametitle{proof of theorem 6.2.3.}

	\begin{itemize}
		\item $f(0) = b$. Let $g = \varphi_b \circ f$. Then $g$ fixes origin.
		\item $g(z) = wz$ for some $|w| = 1$. Namely, $f(z) = \varphi_b^{-1}(wz)$. Note that any $f$ must be such form.
		\item But $\varphi_b^{-1} = \varphi_{-b}$.
		\item $f(z) = \frac{wz + b}{1+ \bar{b}wz}$
		\item Simple calculation leads $f(z) = w \varphi_{-bw^{-1}}(z)$.
		\item Take $a = -bw^{-1}$.

	\end{itemize}
\end{frame}
\begin{frame}
	\frametitle{automorphism group}
	\begin{itemize}
	
		\item Set of all conformal self mapping of unit disc forms group under composition. It is called automorphism group of unit disc.

		\item Mobius transformation denotes automorphism of unit disc.
		\item Further, fix $U$ then $\left\{ \text{conformal self mapping of }U \right\}$ forms a group under composition.
		\item it is called automorphism group of $U$.
	\end{itemize}

\end{frame}
\begin{frame}
	\frametitle{preliminaries of linear fractional transformation}
	\begin{itemize}
		\item Riemann sphere is $\mathbb{C} \cup \infty \cong S^2$ by stereographic projection.
		\item $p_i \rightarrow p_0$ in R-sphere is equivalent to $\pi^{-1}(p_i) \rightarrow \pi^{-1}(p_0)$ in $S^2$.
		\item Note that image of north-pole in $S^2$ under projection is $\infty$.
		\item Also, above definition of limit in extended plane is congruent to definition using metric.
	\end{itemize}

\end{frame}

\begin{frame}
	\frametitle{linear fractional transformation}
	\begin{itemize}
		\item $g : \mathbb{C} \rightarrow \mathbb{C}$ is meromorphic iff $\hat{g} : \mathbb{C} \rightarrow \hat{\mathbb{C}}$ is holomoprhic.
		\item $f: \hat{\mathbb{C}} \rightarrow \hat{\mathbb{C}}$
		\item Let $ad-bc \ne 0$, $a, b, c, d \in \mathbb{C}$. $f(z) = \frac{az + b}{cz+d}$ is called linear fractional transformation if it satisfies two more conditions.
		\item If $c=0$, $f(\infty) = \infty$. In this case, $f$ is a linear map.
		\item If $c\ne 0$, $f(-\frac{d}{c})= \infty$, $f(\infty) = \frac{a}{c}$.
		\item Note that $f(p_i) \rightarrow f(p_0)$ when $p_i \rightarrow p_0$ for all $p_0 \in \mathbb{C} \cup \infty$.
		\item Above says continuity of $f$ on the Riemann sphere.
	\end{itemize}
\end{frame}

\begin{frame}
	\frametitle{linear fractional transformation}
	\begin{itemize}
		\item $[ [a, b], [c, d] ] = A \in GL_2(\mathbb{C})$

		\item $A \cdot z = \frac{az+b}{cz+d}$ is a group action (by simple calculation)

		\item $A^{-1} \cdot (A \cdot z ) = I \cdot z = z$, hence $A \cdot z$ has the inverse, hence bijective

		\item We already know that $A \cdot z$ is continuous on Riemann sphere(by previous frame), hence homeomorphism
	\end{itemize}
\end{frame}

\begin{frame}
	\frametitle{l.f. transformation as conformal self mapping of Riemann sphere}
	\begin{itemize}
		\item When $c = 0$, $g(z) = 1/f(1/z)$ is holomorphic near $0$ (because $f(\infty) = \infty$) hence $f$ is holomorphic at $\infty$.
			
		\item When $c \ne 0$, $h(z) = 1/f(z)$ is holomorphic near $z = -d/c$. \\
			Also $g(z) = 1/f(1/z)$ is holomoprhic near $0$.\\
			Therefore $f$ is holomorphic at $-d/c$ and $\infty$.

		\item In both cases, $f$ is self conformal mapping of the Riemann sphere.
	\end{itemize}
\end{frame}

\begin{frame}
	\frametitle{characterizing conformal self mapping of the Riemann sphere}
	\begin{itemize}
		\item Let $\varphi$ be a conformal self mapping of the Riemann sphere.

		\item If $\varphi$ maps $\infty$ to $\infty$, then $\varphi$ must be linear.

		\item If $\varphi(\infty) = a$, then exists $\psi$: l.f. transformation maps $a$ to $\infty$ \\
			Then $\psi \circ \varphi$ maps $\infty$ to $\infty$.\\
			By above, $\psi \circ \varphi$ must be linear, and by considering $\varphi(z) = \psi^{-1}(\alpha z + \beta)$, we can conclude that $\varphi$ must be l.f. transformation.

		\item Thm 6.3.5 : $f$ is conformal self mapping of the Riemann sphere iff $f$ is l.f. transformation.
	\end{itemize}
\end{frame}

\begin{frame}
\frametitle{geometric property of l.f. transformation}
\begin{itemize}
\item Any line on Riemann sphere can be regarded as circle by stereographic projection. \\

\item	Any l.f. transformation can be represented as composite of translation, dialation, and inversion. \\

\item	Translation and dialation maps circle on Riemann sphere to circle. \\

\item	In fact, inversion maps circle to circle. 
	\end{itemize}
\end{frame}
	\begin{frame}
		\frametitle{inversion maps circle to circle}
		\begin{itemize}
\item	$\alpha(x^2 + y^2) + \beta x + \gamma y + \delta = 0$ represents arbitrary circle on Riemann sphere.\\

\item Let $z = x+iy$ and $w = 1/z = u+iv$. Then simple calculation yields $x = u/(u^2 + v^2)$ and $y = -v/(u^2+v^2)$.

\item $\alpha + \beta u - \gamma v + \delta(u^2 + v^2) = 0$ represents generalized circle also. 

\item therefore we can see that inversion maps circle to circle on Riemann sphere.

\item note that $0 \mapsto \infty$ by inversion. So circle pass through origin goes to line does not pass origin and so forth.
\end{itemize}
\end{frame}

\begin{frame}
	\frametitle{thm 6.3.6.}
	\begin{itemize}
		\item Consider $f:z \mapsto \frac{z-i}{z+i}$.
		\item This maps $(1, 0, \infty, i)$ to $(\frac{1-i}{1+i}=-i, -1, 1, 0)$ respectively.
		\item Note that $1, 0, \infty$ are points in boundary of upper half plane.
		\item And $-i, -1, 1$ are points in boundary of unit disc.
		\item $f$ maps boundary of upper half plane onto boundary of unit disc.
		\item $f$ maps upper half plane onto unit disc since $f$ is continuous and upper half plane is connected.
	\end{itemize}
\end{frame}
\end{document}

