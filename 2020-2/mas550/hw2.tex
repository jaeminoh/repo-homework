\begin{problem}[1.3.1] \hfill

	Since $\sigma\left( X \right)$ is the smallest $\sigma$-field which makes $X$ measurable, it sufficient to show that $X$ is measurable with respect to $\sigma \left ( X^{-1}\left( \mathcal{A} \right )\right)$.

	Let $X:\Omega \rightarrow S$. It is clear that $\left\{ X \in A \right\} \in \sigma\left( X^{-1} \left( \mathcal{A} \right) \right)$ for all $A \in \mathcal{A}$. But by theorem 1.3.1, since $\mathcal{A}$ generates $\mathcal{S}$, $X$ is measurable with respect to $\sigma\left( X^{-1} \left( \mathcal{A} \right) \right)$.  

	Therefore we can conlude that $\sigma\left( X^{-1}\left( \mathcal{A} \right) \right) \subset \sigma\left( X \right)$, and reverse inclusion is canonical since $X^{-1}\left( \mathcal{A} \right) \subset \sigma\left( X \right)$.
\end{problem}

\begin{problem}[1.4.1] \hfill

	Let $E_n = \left\{ x: f(x) > {1\over n} \right\}$. Then $\int f d\mu \geq \int_{E_n} f d\mu \geq \int_{E_n} {1\over n}d\mu = {1\over n}\mu\left( E_n \right)$.
	Therefore $\mu\left( E_n \right) = 0$ for every positive integer $n$. So, $\mu\left( \left\{ f > 0 \right\} \right) = \sum_{n=1}^{\infty}\mu\left( E_n \right) = 0$. This says $f = 0$ a.e.
	
\end{problem}

\begin{problem}[1.4.2]
	Since $E_{n+1, 2m} \cup E_{n+1, 2m+1} = E_{n, m}$ and ${2m+1 \over 2^{n+1}} \ge {m \over 2^{n}}$, we can easily see that $\sum_{m\geq 1}{m \over 2^n} \mu\left( E_{n, m} \right)$ is monotonically increasing as n grows.

	For every positive integer $M$, $\sum_{m = 1}^{M} {m \over 2^n} \mu\left( E_{n, m} \right) \leq \int f d\mu$. So $\sum_{m\geq 1}{m \over 2^n}\mu\left( E_{n, m} \right) \leq \int f d\mu$.

	Let $s_n = \sum_{m=1}^{n 2^n}{m \over 2^n}1_{E_{n, m}}$. Then $\int s_n d\mu \leq \sum_{m \geq 1}{m \over 2^n} \mu\left( E_{n, m} \right) \leq \int f d\mu$. 
	But $s_n \uparrow f$ monotonically. By monotone convergence theorem, $\lim_{n\rightarrow \infty}\int s_n d\mu = \int f d\mu$. Hence by sandwich lemma, the desired result follows.

\end{problem}
