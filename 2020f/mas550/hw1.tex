\begin{problem}[1.1.2] \hfill

	Let $A = \Pi_{i=1}^d \left ( a_i, b_i \right ]$. Then 
	\begin{equation*}
		A = \left ( \Pi_{i=1}^d [a_i -1, b_i] \right ) \cap \left ( \Pi_{I=1}^d (a_i, b_i +1) \right )
		\label{<+label+>}
	\end{equation*}
	which is intersection of open set and closed set.
	So, $A \in \mathcal{R}^d$ therefore $\sigma\left( S_d \right) \subset \mathcal{R}^d$.

	On the other hand, let $B = \Pi_{i=1}^d (a_i, b_i)$ where $-\infty < a_i < b_i < \infty$.
	We can choose sequences $\left\{ a_{i, j} \right\}_{j=1}^{\infty}$ and $\left\{ b_{i, j} \right\}_{j=1}^{\infty}$ for each $1 \leq i \leq d$ such that $a_{i, j} \downarrow a_i$ and $b_{i, j} \uparrow b_i$.
	Then $B_n = \Pi_{i=1}^d (a_{i, n}, b_{i, n}] \uparrow B$. So $B$ is a countable union of open rectangles, hence $B \in \sigma\left( S_d \right)$. 
	Since such $B$ forms basis of topology on $\mathbb{R}^d$, we can conclude that $\mathcal{R}^d \subset \sigma\left( S_d \right)$. 

	
\end{problem}

\begin{problem}[1.2.3] \hfill

	Let $F$ be a distribution function. It is nonnegative, nondecreasing. So $\lim_{y\downarrow x}F(y)$ and $\lim_{y\uparrow x}F(y)$ always exist.
	Let $x$ be a point where $F$ is discontinuous. Since $F$ is discontinuous at $x$, we can assume without loss of generality $\lim_{y\downarrow x}F(y) > F(x)$. Choose a rational number $q_x \in \left ( F(x), \lim_{y\downarrow x} F(y) \right )$. Then function $x \mapsto q_x$ is injective since $F$ is nondecreasing. So there is injection from set of discontinuities to rational numbers. Now we can conclude that set of discontinuities is at most countable.
\end{problem}

\begin{problem}[1.3.4] \hfill
	
	\begin{enumerate}[label = (\alph*)]
		\item Let $f:\mathbb{R}^d \rightarrow \mathbb{R}$ be a continuous function. Consider $\mathcal{B} = \left\{ U \subset \mathbb{R} : f^{-1}\left( U \right) \in \mathcal{R}^d \right\}$. It is well known that $\mathcal{B}$ is a $\sigma$-field. By continuity of $f$, $\mathcal{B}$ contains every open set of $\mathbb{R}$, hence $\mathcal{R} \subset \mathcal{B}$. Therefore $f$ is a measurable function.

		\item Let $\mathcal{F}$ be a $\sigma$-field that makes all the continuous functions measurable.
			Let $\pi_i : \mathbb{R}^d \rightarrow \mathbb{R}$ be the projection on $i$-th factor, which is continuous.
			Then $\cap_{i=1}^{d}\pi_i^{-1}\left( (a_i, b_i) \right) = \Pi_{i=1}^{d}(a_i, b_i) \in \mathcal{F}$. 
			Since $\mathcal{F}$ contains every open rectangles in $\mathbb{R}^d$, we can conclude that $\mathcal{R}^d \subset \mathcal{F}$. This means $\mathcal{R}^d$ is the smallest such $\sigma$-field.
			The fact that $\mathcal{R}^d$ makes all the continuous functions measurable is written in (a).
	\end{enumerate}
	
\end{problem}
