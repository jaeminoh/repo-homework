\section*{chapter 5. Algebras of Sets and Measurable Functions}
\subsection*{section A. algebras and $\sigma$ algebras} \hfill \\

Let $X$ be given set. Algebra $\mathcal{A}$ on $X$ denotes subset of $2^{X}$ satisfying 3 conditions.
\begin{enumerate}
	\item $\emptyset \in \mathcal{A}$ and $X\in \mathcal{A}$
	\item $\cup$ closed.
	\item complement closed.
\end{enumerate}

Similarly, $\sigma$ algebra $\mathcal{F}$ on $X$ denotes subset of power set of $X$, which contains empty set, entire set and is $\sigma -\cup$ closed and complement closed.\\

Problem 9. \\

Clearly, $\mathcal{M}_{0} \subset \mathcal{M}_{1}$. So $\mathcal{M}_{1} \supset \sigma \left (\mathcal{M}_{0} \right ) $ which is the smallest $\sigma$ algebra containing $\mathcal{M}_{0}$.
Let $A \in \mathcal{M}_{1}$. Without loss of generality, we can assume $A$ is countable set. Then $A = \{ a_i : i \in \mathbb{N} \}$ and each $\{ a_i \} \in \mathcal{M}_{0}$. Therefore $A$ is a countable union of sets in $\mathcal{M}_{0}$. Which says $\mathcal{M}_{1} \subset \sigma \left ( \mathcal{M}_{0} \right ) $.\\

Problem 11. \\

Let $y \in A_x$. Then $A_y \subset A_x$. If $x \notin A_y$, $x \in A_x \setminus A_y \in \mathcal{B}$. So $A_x \subset A_x \setminus A_y$ which is contradiction. All other procedure is easy to prove. Just follow the hint. \\
