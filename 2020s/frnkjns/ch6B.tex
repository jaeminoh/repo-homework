\subsection*{section B. General Measurable Functions}\hfill \\

Problem 9. "iff condition for (finite) integrability"\\

Let $f\in \mathcal{L}^1(\mu)$. Then $\int f_{\pm} d\mu < \infty$, so there sum is also finite ( $= \int \lvert f \rvert d\mu$).

Conversely, assume $\lvert f \rvert \in \mathcal{L}^1(\mu)$. Then $\int f_{\pm} d\mu \leq \int \lvert f \rvert d\mu < \infty$. Therefore $f \in \mathcal{L}^1(\mu)$.\\

Problem 10. "dominated integrability"\\

If $f$ is measurable, $f_{\pm}$ is also measurable. So $f_+ + f_- = \lvert f \rvert $ is measurable.
$\int \lvert f \rvert d\mu \leq \int \lvert g \rvert d\mu < \infty$. 
Therefore, $\lvert f \rvert \in \mathcal{L}^1(\mu)$ by Problem 9.\\

Problem 11. \\

It is obvious that $ \lvert f_k \rvert \leq \lvert f \rvert$ and $f \in \mathcal{L}^1(\mu)$.
So, if each $f_k$s are measurable, by dominated convergence thm, done.
Actually, $f_k = f \cdot 1_{A_k \cap E_k}$ where $A_k = \left [ -k, k \right ]$ and $E_k = f^{-1} (\left [ -k, k \right])$.
There exists sequence of nonnegative finite simple function $\{s_i\}$ which converges to $f$ since $f$ is measurable.
So, $\lim s_i 1_{a_k \cap E_k} = f_k$ is measurable.\\

Problem 12. \\

Likewise, it is enough to show that $f(x) e^{-{ {\lvert x \rvert}^2 \over k}} = f_k(x)$ is measurable.
$e^{-{ {\lvert x \rvert}^2 \over k}}$ is continuous, hence Borel measurable, hence Lebesgue measurable.
There exists sequence of nonnegative finite simple function $\{s_k \}$ which converges to $f$ since $f$ is measurable.
So, $\lim_{i \rightarrow \infty} s_i e^{-{ {\lvert x \rvert}^2 \over k}} = f_k$ is measurable. \\

Problem 13. 'alternative proof for problem 2.42' \\

For each $x \in X$, there are at most $d\in \mathbb{N}$ distinct $A_k$ containing $x$.
Fix positive integer $N$. Let $I_x = \{ k \in \mathbb{N} : k \leq N \text{ and } x\in A_k\}$.
Clearly $\sum_{k=1}^N 1_{A_k} \leq d 1_A$ where $A = \bigcup_{i=1}^{\infty}A_i$.
By integrating both sides, $\sum_{k=1}^N \mu(A_k) \leq d \mu(A)$.
By Letting $N \rightarrow \infty$, we got the result in Problem 2.42.

(similar, another) $\sum_{k=1}^N 1_{A_k} \leq \left | I_x \right | \leq d = d 1_A$ for all N.
So $\sum_{k=1}^{\infty} 1_{A_k} \leq d1_A$.
By integrating both sides and monotone convergence thm, we got it.\\

Problem 14. \\

Let $\mathcal{I}$ be set of all sequences which are strictly increasing positive integers and length $m$.
Such set is countable. Now, $\bigcup_{i \in \mathcal{I}} \bigcap_{j=1}^m A_{i_j} = E_m$ and it is measurable.

Similar to Problem 13, $m 1_{E_m} \leq \sum_{i=1}^{\infty}1_{A_i}$.
So, $m \mu(E_m) = \int m 1_{E_m} d\mu \leq \int \sum_{i=1}^{\infty}1_{A_i} d\mu = \sum_{i=1}^{\infty} \int 1_{A_i} d\mu = \sum_{i=1}^{\infty}\mu(A_i)$ by monotone convergence thm.
