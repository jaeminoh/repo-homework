\subsection*{section F. Some Calculations} \hfill \\

Problem 31. \\

$\overline{\mu}\left( A \right) = 0$ implies existence of $B, C \in \mathcal{M}$ such that $B \subset A \subset C$ and $\mu\left( B \right) = \mu\left( C \right) = 0$.
So $\emptyset \subset E \subset C$, $\emptyset, C \in \mathcal{M}$ and $\mu\left( C \right) = 0$. By definition of $\overline{\mathcal{M}}$, $E \in \overline{\mathcal{M}}$ and $\overline{\mu}\left( E \right) = \mu\left( C \right) = 0$.\\

Problem 32. \\

Let $B \in \overline{\mathcal{M}}$. Then there are $A, C \in \mathcal{M}$ such that $A\subset B \subset C$, $\mu\left( C \setminus A \right) = 0$. Let $N = B \setminus A \subset C\setminus A$ which is $\mu -$null set. Then $B = A \cup N$. So every elements in $\overline{\mathcal{M}}$ can be expressed as the form of $A \cup N$ such that $A\in \mathcal{M}$ and $N$ is subset of $\mu$-null set.

On the contrary, consider $A\cup N$, $A\in \mathcal{M}$ and $N$ is subset of $N'$ which is $\mu$-null set.
Then $A \subset A\cup N \subset A \cup N'$ with $\mu\left( A\cup N' \setminus A \right) = 0$, $A, A\cup N' \in \mathcal{M}$. So $A\cup N \in \overline{\mathcal{M}}$. \\

Problem 33. \\

Let $\mathcal{A}$ be an union of $\mathcal{M}$ and collection of subsets of $\mu$-null sets. By problem 32, $\overline{\mathcal{M}} \subset \sigma\left( \mathcal{A} \right) $.
But $\mathcal{A} \subset \overline{\mathcal{M}}$ because $\overline{\mathcal{M}}$ is complete and containing $\mathcal{M}$.

Therefore $\sigma\left( \mathcal{A} \right) = \overline{\mathcal{M}}$. \\

Problem 36. \\

$\{ f > {1 \over k} \}$ must be finite since $\sum_{x\in X}f(x) < \infty$. Therefore $\bigcup \{ f > {1 \over k} \} = \{ f > 0 \}  $ is countable. \\

Problem 37. \\

Let $F$ be finite subset of $\mathbb{N}$. Then $\sum_{x\in F} f(x) \leq \sum_{k=1}^{F_{\text{max} } } f(k) $. Therefore $\sum_{x\in F} f(x) \leq \lim_{n\rightarrow \infty } \sum{k=1}^{n}f(k)$.

Conversely, $\sum_{k=1}^{n}f(k) \leq \sum_{x\in F}f(x) \leq \sum_{x\in \mathbb{N}}f(x)$ for each $n$. \\

Proposition says when counting measure and nonnegative measurable function is given, $\int_{X}f d\mu = \sum_{x\in X}f(x)$. \\

Problem 38. \\

First assume $f \in L^1$. Then $\left| f  \right |$ is nonnegative function. So $\int_{X} \left | f  \right | d\mu = \sum_{x\in X} \left| f(x)  \right | < \infty$ by proposition above.

Conversely, $\sum_{x\in X} \left | f(x)  \right | = \int_{X} \left| f  \right | d\mu < \infty$. Therefore $\int_{X} f^{\pm} d\mu \leq \int_{X}\left | f  \right | d\mu < \infty$. So $f \in L^1$.\\


