\section*{chater 7: Lebesgue Integral on $\mathbb{R}^n$}

\subsection*{section A: Riemann Integral} \hfill \\

Problem 1. \\

Suppose $1_A$ is LSC. Let $x \in A$ and $0 < 1_A(x) = 1$. By definition of LSC at $x$, there exists $\delta>0$ such that $1_A(y) > 0$ for all $y \in B\left( x ; \delta \right) $. So $1_A(y) = 1$ and therefore $B\left( x ; \delta \right) \subset A$. Thus $A$ is open.

On the contrary, suppose $A$ is open. For $x \in A$, consider $t < 1 = 1_A(x)$. Since $A$ is open, there exists $\delta>0$ such that $B\left( x;\delta \right) \subset A$. Then we get $y\in B\left( x;\delta \right) \Rightarrow t < 1_A(y) = 1$.
For $x \notin A$, consider $t<0 = 1_A(x)$. Take any $\delta > 0$. Then $y \in B\left( x ; \delta \right) \Rightarrow t < 0 \leq 1_A(y)$. Therefore $1_A$ is LSC if $A$ is open.\\

Note that $f$ is LSC if and only if $\forall t \in \bar{\mathbb{R}}$, $\left\{ f > t \right\}$ is open.
And $f$ is LSC at $x$ if and only if $x$ is interior point of every $\left\{ f>t \right\}$ for $t < f(x)$.\\

Problem 2. \\

For $x \in A^\circ$, there exists $\delta > 0$ such that $B\left( x ; \delta \right) \subset A$. Then $\inf_{y \in B\left( x ; \delta \right)}1_A(y) = 1$ so lower envelope of $1_A$ at $x$ is same as $1_{A^\circ}(x)=1$.

Now assume $x \notin A^\circ$. Then, for every $\delta>0$ $B\left( x;\delta \right) \cap A \ne \emptyset$. Then $\inf_{y \in B\left( x;\delta \right)}1_A(y) = 0$ for some small $\delta$. Then lower envelope of $1_A$ is zero at $x$, which is same as $1_{A^\circ}(x)$.\\

If $x \in \bar{A}$, for all $\delta >0$ $B\left( x;\delta \right) \cap A \ne \emptyset$. Therefore $\sup_{y\in B\left( x;\delta \right)}1_A(y) = 1$, so upper envelope of $1_A$ is same as $1_{\bar{A}}$.

If $x \notin \bar{A}$, there exists $\delta>0$ such that $B\left( x;\delta \right) \cap A = \emptyset$. Then $\sup_{y\in B\left( x ; \delta \right)}1_A(y) = 0$. So upper envelope of $1_A$ is same as $1_{\bar{A}}$.\\

Problem 3. \\

For each $x \in \mathbb{R}^n$, let $t < \min_{i}f_i(x)$. Then $t<f_i(x)$ for all $i \in \mathcal{I}$. For each $i$, there exists $\delta_i$ such that $y\in B\left( x;\delta_i \right)$ implies $t < f_i(y)$. Take $\delta = \max_{i}\delta_i$. Then $y \in B\left( x;\delta \right)$ implies $t <\min_{i} f_i(y)$. So $\min_{i}f_i $ is LSC.

Now consider $A_i = \left(- \frac{1}{i}, \frac{1}{i} \right)\subset \mathbb{R}$. Since $A_i$ is open, by problem 1, $1_{A_i}$ is LSC. Let $A = \bigcap_{i=1}^{\infty}A_i$ then $1_A = \inf_{i}1_{A_i}$. It is not semicontinuous by considering the set $\left\{ 1_A > \frac{1}{2} \right\}=\left\{ 0 \right\}$. \\

Problem 4. \\

Let $\tau_f \geq f$ and $\tau_g \geq g$ where $\tau_f$ and $\tau_g$ are step functions. Note that $\tau_f + \tau_g$ is also step function greater than $f+g$. Then all others follow directly. \\

Problem 5. \\

Let $\varepsilon>0$ be given. We can choose positive integer $N$ such that 

\begin{enumerate}
	\item $\left | f(x) - f_n(x) \right | < \varepsilon$ if $n \geq N$ and for all $x \in \mathbb{R}^n = X$.
	\item $\int_I\left ( \tau_N - \sigma_N\right ) d\lambda < \varepsilon$
\end{enumerate}

where $\tau_N, \sigma_N$ is simple function bigger, smaller than $f_N$ respectively.

For every $x\in X$,
\begin{equation*}
	\sigma_N(x) - \varepsilon \leq f_N(x) -\varepsilon < f < f_N(x) +\varepsilon \leq \tau_N(x) + \varepsilon
	\label{2}
\end{equation*}

Then, $\int_I \sigma_N d\lambda -\varepsilon \lambda\left( I \right) \leq r\underline{\int_I}fd\lambda \leq r\overline{\int_I}fd\lambda \leq \int_I \tau_N d\lambda + \varepsilon\lambda\left( I \right)$
Because $\sigma_N -\varepsilon$ is step function smaller than $f$ and $\tau_N + \varepsilon$ is also step function bigger than $f$. 

Therefore we can get
\begin{equation*}
	r \overline{\int}_I fd\lambda - r\underline{\int}_I f\lambda < \varepsilon + 2\varepsilon \lambda\left( I \right)
	\label{3}
\end{equation*}
which implies Riemann integrability of $f$.

By definition of uniform convergence, $r\int \left | f - f_N \right | d\lambda \leq \varepsilon \lambda\left( I \right)$.\\
So, $\lim_{N\rightarrow \infty}r\int \left | f - f_N \right | d\lambda = 0$, which implies $\lim_{N\rightarrow \infty}\left | r\int fd\lambda - r\int f_N d\lambda \right | = 0$ then conclusion follows. \\

Problem 6.\\

\begin{enumerate}[label = (\alph*)]
	\item $g(x) = 0$ for all $x \in I = \left[ 0, 1 \right]$. $f(x) = 1_{\mathbb{Q}\cap I}(x)$. $f$ is nowhere continuous but $f = g$ almost everywhere.
	\item Consider the following funnction $f$ :
		\begin{equation*}
			f(x) = 
			\begin{cases}
				0 & \text{ if } x \in I\cap C \\
				{1 \over x } & \text{ if } x \in I \cap C^{c}
			\end{cases}
			\label{<+label+>}
		\end{equation*}
		where $C$ is Cantor ternary set and $I = \left[ 0, 1 \right]$.
		
		For $x \in I \cap C^c$, $x \in I_{j, k} =  \left( {2k \over 3^j}, {2k+1 \over 3^j} \right)$ for some $k, j$. 
		On $I_{j, k}$, function $i : x \mapsto x$ is continuous and $i(x) \ne 0$. So, the map $\varphi:x \mapsto {1\over x}$ is also continuous on $I_{j, k}$.
		Therefore, $\varphi$ is continuous at $x$. Therefore, $\varphi$ is continuous a.e. on $I$.

		Note that $f(x) = {1\over x }$ a.e. on $I$. Let $g = f$ a.e. on $I$. Then $g(x) = {1\over x}$ a.e. on $I$. Then $g$ is discontinuous at $x = 0$. Therefore, $f$ is continuous a.e. on $I$ and there is no continuous function such that $f = g$ a.e. on $I$.
\end{enumerate}

\hfill \\

Problem 7. \\

For $x \in \left[ a, b \right]$, $na \leq nx \leq nb$, so $nx \in {1 \over 2} + \mathbb{Z}$ for finitely many $x$. Therefore, $(nx)$ is discontinuous at most finitely many points, which implies $\left( nx \right)$ is Riemann integrable.
Then $\frac{\left( nx \right)}{n^2}$ also Riemann integrable, and their finite summation $f_k (x) = \sum_{n=1}^{k} \frac{\left( nx \right)}{n^2}$ is also Riemann integrable.

Now, let $\varepsilon > 0$ be given. Since $\sum_{n=1}^{\infty}$ converges, there exists positive integer $N$ such that $\sum_{k=N}^{\infty} < \varepsilon$.
Consider $m > n \geq N$ and following:
\begin{equation*}
	\left | f_m(x) - f_n(x)  \right | \leq \left | \sum_{k=n+1}^{m} \frac{\left( kx \right)}{k^2} \right |
	\leq \sum_{k=n+1}^{m}\frac{\left | \left( kx \right) \right |}{k^2} \leq \sum_{k=n+1}^{m} \frac{1}{k^2} < \varepsilon
	\label{<+label+>}
\end{equation*}
for all $x \in \left[ a, b \right]$ since $ -1 \leq \left( nx \right) \leq 1$.
Therefore, $f_n$ is uniformly Cauchy, which implies uniform convergence of $f_n$ to $f$.
By problem 5, $f$ is Riemann integrable since each $f_n$ is Riemann integrable and $f_n \rightrightarrows f$. \\

Problem 9. \\

With out loss of generality, assume that $f$ is monotonically increasing.\\

Let $\varepsilon >0$ be given. Consider $a = x_0 < x_1 < \cdots < x_n = b$ \\
where $\max_{ 1\leq k \leq n }\lambda\left( \left[ x_{k-1}, x_k \right] \right) < \frac{\varepsilon}{f(b) - f(a)}$. (If $f(a) = f(b)$, conclusion follows trivially so let us assume that $f(a) < f(b)$.

	Let $I = \left[ a, b \right]$ and $\sigma : I \rightarrow \mathbb{R}$ such that $\sigma\left( \left( x_{k-1}, x_k \right) \right) = \left\{ f(x_{k-1}) \right\} $, $\sigma(x_k) = f(a)$. Similarly, let $\tau: I \rightarrow \mathbb{R}$ such that $\tau\left( \left( x_{k-1}, x_k \right) \right) = \left\{ f\left( x_k \right) \right\}$ and $\tau(x_k) = f(b)$. 
	Then $\sigma, \tau$ are step functions satisfying $\sigma \leq f \leq \tau$.
	So, 
	\begin{equation*}
			\begin{split}
				\int_{I}\left( \tau - \sigma \right) d\lambda = \sum_{k=1}^{n}\left( f(x_k) - f(x_{k-1}) \right)\lambda\left( \left[ x_{k-1}, x_k \right] \right)\\
				< \sum_{k=1}^{n} \left( f(x_k) - f(x_{k-1}) \right) \frac{\varepsilon}{f(b) - f(a)} = \varepsilon
			\end{split}
		\label{<+label+>}
	\end{equation*}
	which implies Riemann integrability of $f$ on $I$.\\

Let $x< x'$ be points where $f$ is discontinuous.
Since $f$ is monotonic, $f(x-) = \lim_{y\uparrow x}f(y)$ and $f(x+) = \lim_{y\downarrow x}f(y)$ exist.
By monotonicity of $f$, we can easily deduce that $f(x-) \leq f(x) \leq f(x+) \leq f(x'-) \leq f(x') \leq f(x'+)$.
Since $f$ is discontinuous at $x, x'$, $f(t-) < f(t+) $ for $t = x, x'$. Choose $q_t \in (f(t-), f(t+))$ for $t = x, x'$. Then $q_x < q_{x'}$. The map $x \mapsto q_x$ is hence injective. So, there are at most countably many discontinuous points of $f$ on $I$.
	
Therefore $f$ is continuous a.e. on $I$.\\

Problem 10.\\

Consider $1_C$ where $C$ is Cantor ternary set.
If $x \in C$, $x \notin C^\circ$ since $C$ has empty interior. So, for any $\delta >0$, there exists $y \in B(x;\delta)$ such that $y \notin C$. Then $1_C(x) - 1_C(y) = 1$. So $1_C$ is discontinuous at $x \in C$.

On the contrary if $x \in C^c$, $x \in I_{j, k}$ for some $j, k$ (we'll use notation of problem 6.)
Then there is $\delta > 0 $ such that $B(x;\delta) \subset I_{j, k} \subset C^c$, so $d(x, y) < \delta$ implies $1_C(x) - 1_C(y) = 0 < \varepsilon$ for any $\varepsilon >0$. Therefore $1_C$ is continuous on $C^c$.

$1_C$ has an uncountable set of discontinuities($C$) and continuous a.e. on $\left[ 0, 1 \right]$. Therefore $1_C$ is Riemann integrable.\\

Problem 11. \\

If $f$ is Riemann integrable and $f = 1_A$ a.e. on $I = \left[ 0, 1 \right]$, $r\int f d\lambda = \int _I f d\lambda = \int_I 1_A d\lambda = \lambda\left( A \right) > 0$.

Let $\sigma$ be a step function, $\sigma \leq f$. Then $\sigma \leq 1_A$ a.e. on $I$ and let $N$ be corresponding null set.

Let $0 = x_0 < x_1 < \cdots < x_n = 1$ be endpoints of special rectangles corresponding to $\sigma$.
If $\left( x_{k-1}, x_k \right) \cap J_{i} = \emptyset$ for all positive $i = \frac{2m-1}{2^t} < 1$ ($ A = I \setminus \bigcup J_i$, we'll use notation of section 4.B), $\left( x_{k-1}, x_k \right) \subset A$ which contradicts the fact that $A$ has empty interior. So, $\left( x_{k-1}, x_k \right) \cap J_i \ne \emptyset$ for some $i$.

If $\left( x_{k-1}, x_k \right) \setminus N \subset A$, $\left( x_{k-1}, x_k \right) \setminus N \cap J_i = \emptyset$, then $\left( x_{k-1}, x_k \right) \cap J_i \subset N$. But $J_i \cap \left( x_{k-1}, x_k \right)$ is open and nonempty, so it has positive measure which contradicts $\lambda(N) = 0$. 

Therefore for each $k$, $\left( x_{k-1}, x_k \right) \setminus N \cap A^c \ne \emptyset$. So $\sigma\left( \left( x_{k-1}, x_k \right) \right) = {\xi}$ where $\xi \leq 0$. It means $\sigma \leq 0$.

Then $r\int f d\lambda = \sup_{\sigma \leq f} \int_I \sigma d\lambda \leq 0$ which contradicts to $r\int fd\lambda = \lambda\left( A \right) >0$.
