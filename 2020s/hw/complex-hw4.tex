\begin{problem}
	Obviously $\left | {1 \over z } \right | \leq {1\over \varepsilon} $. Let $ {1 \over z } = r e^{i\theta} $ where $0 < r \leq {1 \over \varepsilon}$. Then real part of ${1\over z} = r \cos\theta$. Therefore $\left | e^{1 \over z} \right | = e^{r\cos \theta} \leq e^{1 \over \varepsilon}$. Thus, given function is bounded in the region $\left | z  \right | \geq \varepsilon $.
\end{problem}

\begin{problem}
	Let $z = \log\left( 1 + i  \right) ^{\pi i}$. Then $e^{z} = \left( 1 + i  \right) ^{\pi i} = e^{\pi i \log \left( 1 + i \right)}$. Therefore $z = 2k\pi i + \pi i \log \left ( 1 + i \right)$ $ = 2k \pi i + \pi i \left(  \ln \sqrt{2} + i \left(  {\pi\over 4} + 2m \pi  \right) \right)$. So $z = \left( 2k + {1 \over 2 }\ln{2} \right)\pi i - \pi ^{2}\left( {1 \over 4} + 2m \right)$ where $k, m$ are integers.
\end{problem}

\begin{problem}
	\begin{multline*}
	\left( iy  \right)^{\left( iy \right)}
	= e^{iy \log \left( iy \right)}
	= e^{iy \left(  ln \lvert y \rvert + i\left( {\pi \over 2} + k\pi  \right) \right)}\\
	= e^{-y \left( { \pi \over 2 + k\pi } + iy\ln \lvert y \rvert  \right)}
	= e^{-y\left( {\pi \over 2 } + k\pi  \right)} \left(  \cos\left( y\ln \lvert y \rvert  \right) + i\sin \left( y\ln \lvert y \rvert  \right) \right)
	\end{multline*}
where $k$ is even(odd) when $y>0$($y<0$).
\end{problem}

\begin{problem}
	$z = re^{i \theta } $.
	Then $z^{2\over 3} = e^{ {2 \over 3} \log z  } = e^{ {2 \over 3 } \left( \ln r + i \theta  \right)} = r^{2\over 3}\cdot e^{i \left( {2 \over 3 } \theta  \right)} = w$.
	Therefore $\left | w  \right | = r^{2\over 3}$.
\end{problem}

\begin{problem}
	Let $a = {1 \over 2i } \log \frac{z+i}{z-i}$. Then $\sin a = \frac{e^{ai} - e^{-ai}}{2i}$ and $\cos a = \frac{e^{ai}+e^{-ai}}{2}$. 
\begin{multline*}
	\cot a = i \frac{e^{ai}+e^{-ai}}{e^{ai}-e^{-ai}} = i \frac{e^{2ai}+e^{-2ai}+2}{e^{2ai}-e^{-2ai}} \\
	=i \frac{ \frac{ z+i }{z-i} + \frac{z-i}{z+i} + 2 }{\frac{z+i}{z-i} - \frac{z-i}{z+i}} = i \frac{4z^{2}}{4zi} = z
\end{multline*}
Therefore $\cot a = z$ which implies $\cot^{-1 } z = a$.
\end{problem}

\begin{problem}
	\hfill \\
	\begin{enumerate}
		\item $\lim_{z \rightarrow 0 } \frac{\log \left( 1 + z  \right) }{z} = 1$ by theorem 5.26. By continuity of $e^{z}$, $\lim_{z \rightarrow 0} \left( 1+z  \right) ^{1 \over z } = \lim_{z \rightarrow 0} e^{\log \left( 1+z  \right) \over z} = e$.
		\item $z^{1 \over 2 } = e^{ {1 \over 2 } \ln \lvert z \rvert + i \theta}$ where $\theta$ is argument of $z$.
			So, $z^{1\over 2} \rightarrow 0$ as $z \rightarrow 0$.
			Therefore, $\lim_{z\rightarrow 0} \frac{\sqrt{z} - 2}{z-2} = {-2 \over -2 } = 1$.
	\end{enumerate}
\end{problem}

\begin{problem}
	$u_x = \sqrt{x^2 + y^2 } +\frac{x^2}{\sqrt{x^2 + y^2}}$, $u_y = \frac{xy}{\sqrt{x^2 + y^2}}$, $v_x = \frac{xy}{\sqrt{x^2 + y^2}}$, and $v_y = \sqrt{x^2 + y^2} + \frac{y^2 }{ \sqrt{x^2 + y^2 }}$. To satisfy Cauchy-Riemann equation, $x^2 = y^2$ and $xy = 0$ which is impossible except for $x = y = 0$.
	
	Let's compute values of partials.
	$u_x\left( 0, 0 \right) = \lim_{h \rightarrow 0 } \frac{h\sqrt{h^{2}}}{h} = \lim_{h \rightarrow 0 } \lvert h \rvert = 0$. Similarly all other partials have $0$ at the origin. So given function satisfies Cauchy-Riemann equation at origin. So $f$ is differentiable at origin and $f'(0) = 0$.
\end{problem}

\begin{problem}
	$f(z) = z^{2} + z^{2} \bar{z}$. So $\frac{\partial f}{\partial \bar{z}} = z^{2}$ which is zero if and only if $z = 0$. So $f$ is nowhere differentible except at $z = 0$. $\lim_{z\rightarrow 0} \frac{f(z) - f(0)}{z} = \lim_{z \rightarrow 0} \left( z + z \bar{z} \right) = 0 = f'(0)$. 
	Also, we cannot say about derivative of $f$ on $\mathbb{C} - 0$. So we cannot say about second(or higher) order derivative of $f$ at origin.
\end{problem}

\begin{problem}
	$u_x = e^{x} \cos \left( ay \right) $, $u_y = -a e^{x} \sin \left( ay \right)$, $v_x = e^{x}\sin \left( y+b \right)$, and $v_y = e^{x} \cos \left( y+b \right)$. To satisfy Cauchy-Riemann equation, $a = \pm 1$ and $b = 2k\pi$ where $k\in \mathbb{Z}$.	
\end{problem}

\begin{problem}
	Since real part of $f' = f_x$ is zero, $u_x = 0 = v_y$ by Cauchy-Riemann equation. So $u = \phi_1 (y)$ and $v = \phi_2 (x)$. Then $\phi _1 ' (y) = - \phi _2 ' (x)$ for all $x, y \in \mathbb{R}$ because $u_y = -v_x$. Therefore $\phi_1 ' = - \phi_2 ' = c \in \mathbb{R}$. So, put $\phi_1 = cy+d_1$ and $\phi_2 = -cx + d_2$ where $d_1, d_2 \in \mathbb{R}$. Therefore $f(x, y) = (cy+d_1) + i(-cx + d_2)$ which is entire function.
\end{problem}

\begin{problem}
	Automatically, $u_y = -v_x = 0$. 
	$u'(x) = v'(y)$ for all $x, y \in \mathbb{R}$. So $u' = v' = c \in \mathbb{R}$.
	So $u(x) = cx + d_1$ and $v(y) = cy + d_2$ where $d_1, d_2 \in \mathbb{R}$.
	Therefore $f(z) = cz + d_1 + i d_2$.

\end{problem}

\begin{problem}
	Given function is real function. But analytic real valued function must be constant in the domain.
	Unfortunately, our function is not constant. So it is not analytic in the domain.
\end{problem}

\begin{problem}
	$f(z) = \ln \lvert z \rvert + i \theta$ where $4\pi < \theta \leq 6\pi$ is argument of $z$. (that is, cutting $\theta = 4\pi$)

	Clearly, $f$ is analytic on $\mathbb{C} \setminus [0, \infty )$ and $f(-1) = 0 + i \theta$.
	$\theta = \pi + 2k \pi \in (4\pi , 6\pi]$. So $k = 2$ and $f(-1) = 5\pi i$.
\end{problem}
