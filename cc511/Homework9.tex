\documentclass[a4paper,10pt]{article}
\usepackage{titlesec}
\usepackage[centertags]{amsmath}
\usepackage{amsmath}
\usepackage[british]{babel}
\usepackage{amsfonts}
\usepackage{amssymb}
\usepackage{amsthm}
\usepackage{graphicx}
\usepackage{subfig}
\usepackage{newlfont}
\usepackage{color}
\usepackage{float}
\usepackage{bbm}

\usepackage{enumitem}

\usepackage{stmaryrd}
\usepackage{mathrsfs}
\usepackage{pstricks,pst-node}

\usepackage{fancyhdr}
\usepackage{graphicx}
\usepackage{fancybox}
%\usepackage{multibox}
\usepackage{setspace}
\usepackage{hyperref,cleveref}

\usepackage{authblk}
\newtheorem*{sol}{Solution}
%%%%%%%%%%%%%%%%%%%%%%%%%%%%%%%%%%%%%%%%%%%%%%%%%%%%%%%%%%%%%%%%%%%%%%%%%%%%%%%%%%%%%

\begin{document}
\section*{Homework 9.}
(Due 6 PM on May. 19)

\begin{itemize}
\item [10.1.4]  A new radar system is being developed to detect packages
dropped by airplane. In a series of trials, the radar
detected the packages being dropped 35 times out of 44. Construct a 95\% lower confidence bound on the
probability that the radar successfully detects dropped
packages. \\

\begin{sol}
	Since $n-x = 9  > 5$ and $n = 44 > 5$, we can use normal approximation. (+3)
	The formula for the answer is
	\[
		\left (\frac{x}{n} - \frac{Z_{0.05}}{n}\sqrt{ \frac{n(n-x)}{n}}, 1 \right ),
	\]
	which is
	\[
		\left( \frac{35}{44} - \frac{1.645}{44}\sqrt{\frac{35 \cdot 9}{44}}, 1 \right) = (0.70, 1).
	\](+7)

	\qed
\end{sol}

\item [10.1.8] In trials of a medical screening test for a particular
illness, 23 cases out of 324 positive results turned out to
be ``false-positive'' results. The screening test is
acceptable as long as $p$, the probability of a positive
result being incorrect, is no larger than 10\%. Calculate a
$p$-value for the hypotheses
\begin{align*}
	&&H_{0}: p \geq 0.1 && \text{versus}&& H_{A}:p<0.1&&
\end{align*}
Construct a 99\% upper confidence bound on $p$. Do you
think that the screening test is acceptable?

\begin{sol}
	$n-x, n > 5$ so the normal approximation can be done. (+2)
	Our observed statistic is
	\[
		\frac{x-np_0 + 0.5}{\sqrt{np_0(1-p_0)}} = -1.65
	\]
	where $+0.5$ of the numerator is due to the continuity correction.(+5)

	Now, p-value is
	\[
		P(Z \le -1.65) = 0.049.
	\](+2)

	And the upper confidence bound is
	\[
		\left( 0, \frac{x}{n} + \frac{Z_{0.01}}{n}\sqrt{\frac{x(n-x)}{n}} \right) = (0, 0.104).
	\](+1)

	Since p-value of our test is larger than $0.01$, we can say that the screening thest is non-acceptable.

	\qed
\end{sol}

\item [10.1.18] The dielectric breakdown strength of an electrical
insulator is defined to be the voltage at which the
insulator starts to leak detectable amounts of electrical
current, and it is an important safety consideration. In
an experiment, 62 insulators of a certain type were
tested at $180^{\circ}$C, and it was found that 13 had a
dielectric breakdown strength below a specified
threshold level.
\begin{enumerate}
	\item [(a)] Conduct a hypothesis test to investigate whether this
	experiment provides sufficient evidence to conclude
	that the probability of an insulator of this type
	having a dielectric breakdown strength below the
	specified threshold level is larger than 5\%.
	\item [(b)]Construct a one-sided 95\% confidence interval that
	provides a lower bound on the probability of an
	insulator of this type having a dielectric breakdown
	strength below the specified threshold level.
\end{enumerate}
\begin{sol}
	\begin{enumerate}[label = (\alph*)]
		\item $n= 62$, $x = 13$ and $p_0 = 0.05$.
			Note that the normal approximation is applicable for our data.
			Set the null hypothesis $H_0: p \le 0.05$ and the alternative $H_1: p > 0.05$.
			Now observed test statistic is
			\[
				\frac{x- np_0 - 0.5}{\sqrt{np_0(1-p_0)}} = 5.48
			\]
			by normal approximation and continuity correction.
			Then p-value is
			\[
				P(Z > 5.48),
			\]
			which is quite smaller than usual confidence level($0.05, 0.01$).
			So we can reject the null under $\alpha = 0.01, 0.05$.(+5)

		\item By using the same formula which is used in the previous problem,
			the answer is
			\[
				\left( \frac{13}{62} - \frac{Z_{0.01}}{62}\sqrt{\frac{13(62-13)}{62}}, 1 \right)
				 = (0.125, 1).
			 \](+5)
	\end{enumerate}
	
	\qed
\end{sol}
\item [10.2.2] Suppose that $x = 261$ is an observation from a
$B(302, p_A)$ random variable, and that $y = 401$ is an
observation from a $B(454, p_B)$ random variable.
\begin{enumerate}
	\item [(a)] Compute a two-sided 99\% confidence interval for
	$p_A-p_B$.
	\item [(b)] Compute a two-sided 90\% confidence interval for
	$p_A-p_B$.
	\item [(c)]Compute a one-sided 95\% confidence interval for
	$p_A-p_B$. Repeat (a) and (b) with the data in (c). 
	\item [(d)]	Calculate the $p$-value for the test of the hypotheses		
	\begin{align*}
		&& H_{0}: p_A=p_B && \text{versus} && H_{A}:p_{A}\neq p_{B}
	\end{align*}
\end{enumerate}
\begin{sol}
	$(n, m, x, y) = (302, 454, 261, 401)$.
	\begin{enumerate}[label = (\alph*)]
		\item The formula is
			\[
				\left( \hat{p}_A - \hat{p}_B - Z_{\alpha/2}\sqrt{\frac{\hat{p}_A (1-\hat{p}_A)}{n} + \frac{\hat{p}_B(1-\hat{p}_B)}{m}}, \hat{p}_A - \hat{p}_B + Z_{\alpha/2}\sqrt{\frac{\hat{p}_A(1-\hat{p}_A)}{n}+ \frac{\hat{p}(1-\hat{p}_B)}{m}} \right).
			\]
			Realized value is $(-0.083, 0.045)$.(+2)

		\item Procedure is the same. (+2)

		\item Procedure is the same. (+2)

		\item Our test statistic is
			\[
				z = \frac{\hat{p}_A - \hat{p}_B}{\sqrt{\hat{p}(1-\hat{p})\left( 1/n + 1/m \right)}} = -0.78
			\]
			where $\hat{p}$ is the pooled one.
			Since we are doing two-sided test,
			our p-value is
			\[
				P(|Z| \ge 0.78) = 0.44.
			\](+4)
	\end{enumerate}

	\qed
\end{sol}
\item [10.2.12] Recall from Problem 10.1.16 that in a particular day, 22
out of 542 visitors to a website followed a link provided
by an advertiser. After the advertisements were
modified, it was found that 64 out of 601 visitors to the
website on a day followed the link. Is there any evidence
that the modifications to the advertisements attracted
more customers? ( $\alpha=0.05$)

\begin{sol}
	$(n, m, x, y) = (542, 601, 22, 64)$.
	Set the null $H_0: p_A \ge p_B$ and the alternative $H_1: p_A < p_B$.(+3)
	Use test statistic $z$ used in the above.
	Then observed test statistic is $-4.22$.
	So, p-value is pretty smaller than 0.05 since p-value is $P(Z \le -4.22)$, and $-4.22 < -1.64$.
	Thus we can reject the null, so there is evidence of the claim.(+7)
	
	\qed
\end{sol}
\item [\textcolor{red}{10.1.16}] [\textcolor{red}{Not a homework problem}]
In a particular day, 22 out of 542 visitors to a website
followed a link provided by one of the advertisers.
Calculate a 99\% two-sided confidence interval for the
probability that a user of the website will follow a link
provided by an advertiser.

\item [10.3.6]\textbf{Taste Tests for Soft Drink Formulations}\\
A beverage company has three formulations of a soft
drink product. DS 10.3.6 gives the results of some taste
tests where participants are asked to declare which
formulation they like best. Is it plausible that the three
formulations are equally popular? ( $\alpha=0.05$)

\begin{sol}
	Since we are using the approximated statistic, checking the approximation condition is always important.
	And note that $\chi^2, G^2$ are asymptotically chi-squared.
	So if $n$ is sufficiently large, then you can use anything.
	I'll use the first one.

	Set the null $H_0: p_i = 1/3, i= 1,2,3$ and the alternative $H_1:\text{ not }H_0$.(+3)
	The total number of participants is $600$, so the expected frequencies are $e_i = 200$ each.(+3)
	Our test statistic is
	\[
		\sum_{i=1}^3\frac{(x_i - e_i)^2}{e_i} = 17.29
	\]
	where $x_i$ is the frequency of each cell.(+2)
	Now p-value is
	\[
		P(\chi^2_2 \ge 17.29) = 0.00018
	\]
	so we can reject the null under $\alpha = 0.05$.(+2)

	\qed
\end{sol}
\item [10.3.14] A survey is performed to test the claim that three brands
of a product are equally popular. In the survey, 22 people
preferred brand A, 38 people preferred brand B, while
40 people preferred brand C. The Pearson goodness
of fit can be used to test whether the survey provides
sufficient evidence to conclude that the three brands are
not equally popular. ( $\alpha=0.05$)

\begin{sol}
	The total number of participants is $100$.
	Let $(x_1, x_2, x_3) = (22, 38, 40)$.
	Set the null $H_0: p_i = 1/3$ and the alternative $H_1: \text{ not }H_0$.(+3)
	Then the expected frequency is $100/3 = e_i$ each.(+3)
	Our test statistic is
	\[
		\sum_{i=1}^3 \frac{\left( x_i - e_i \right)^2}{e_i} = 5.84.
	\](+2)
	So, p-value is
	\[
		P(\chi^2_2 \ge 5.84) = 0.054,
	\]
	thus we cannot reject the null under $\alpha = 0.05$.(+2)
	(False: the survey does not provide sufficient evidence.)
	
	\qed
\end{sol}
\begin{align*}
	\text{A. } \text{True} && \text{B .} \text{False} && &&
\end{align*}
\item [10.4.2]\textbf{Fertilizer Comparisons}\\
Seedlings are grown without fertilizer or with one of two
kinds of fertilizer. After a certain period of time a
seedling’s growth is classified into one of four
categories, as given in DS 10.4.2. Test whether the
seedlings’ growth can be taken to be the same for all
three sets of growing conditions. ($\alpha=0.05$)

\begin{sol}
	Sort the data by the matrix($ = A$), where the same row indicates fertilizer and the same column indicates growth.
	(row1 = no fer, row2 = fer1, row3 = fer2, col1 = dead, col2 = slow, \dots)

	Set the null $H_0: \text{ two categories are independent}$ and the alternative $H_1: \text{ not }H_0$.(+3)
	Let $x_{ij} = (A)_{i, j}$ be the $(i, j)$ entry of our matrix $A$.
	Then expected frequency is
	\[
		e_{ij} = \frac{x_{i \cdot}x_{\cdot j}}{n}
	\]
	where $x_{i\cdot} = \sum_{j=1}^4 x_{ij}$.(+3)
	
	Then our test statistic is
	\[
		X^2 = \sum_{i=1}^3 \sum_{j=1}^4 \frac{(x_{ij} - e_{ij})^2}{e_{ij}} = 13.66.
	\](+2)
	So p-value is
	\[
		P(\chi^2_6 \ge 13.66) = 0.034.
	\]
	Thus we can reject the null.(+2)
	
	\qed
\end{sol}
\item [10.4.6] Show that for a $2\times 2$ contingency table the Pearson
chi-square statistic can be written
$$
X^{2}=\frac{n(x_{11}x_{22}-x_{12}x_{21})^{2}}{x_{1\cdot}x_{\cdot 1}x_{2\cdot}x_{\cdot 2}}
$$

\begin{sol}	
	Note that
	\[
		x_{ij} - e_{ij} = x_{ij} - \frac{(x_{i1} + x_{i2})(x_{1j} + x_{2j})}{\sum_{i, j}x_{ij}}
		= \frac{x_{ij}x_{i'j'} - x_{ij'}x_{i'j}}{n}
		= \frac{x_{11}x_{22} - x_{12}x_{21}}{n}
	\]
	where $i \ne i'$ and $j \ne j'$.

	Thus
	\[
		\sum_{i, j}\frac{(x_{ij}-e_{ij})^2}{e_{ij}} = \frac{(x_{11}x_{22} - x_{12}x_{21})^2}{n^2}\sum_{i, j}\frac{1}{e_{ij}}.
	\]
	But
	\[
		\frac{1}{e_{ij}} = \frac{n}{x_{i\cdot}x_{\cdot j}} = \frac{n x_{i'\cdot}x_{\cdot j'}}{x_{1\cdot}x_{\cdot 1}x_{2\cdot}x_{\cdot 2}}.
	\]
	Also,
	\[
		\sum_{i, j}x_{i\cdot}x_{\cdot j} = n^2.
	\]
	Combinining the above,
	\[
		X^2 = \frac{(x_{11}x_{22} - x_{12}x_{21})^2}{n^2} \frac{n}{x_{1\cdot}x_{\cdot 1}x_{2\cdot} x_{\cdot 2}} n^2.
	\](+10)

	\qed
\end{sol}

\item [10.4.10]\textbf{Asphalt Load Testing}\\
An experiment was conducted to compare three types of
asphalt. Samples of each type of asphalt were subjected
to repeated loads at high temperatures, and the resulting
cracking was analyzed. For type A, 57 samples were
tested, of which 9 had severe cracking, 17 had medium
cracking, and 31 had minor cracking. For type B, 49 samples were tested, of which 4 had severe cracking,
9 had medium cracking, and 36 had minor cracking. For
type C, 90 samples were tested, of which 15 had severe
cracking, 19 had medium cracking, and 56 had minor
cracking. Does this experiment provide any evidence
that the three types of asphalt are different with respect
to cracking?
\end{itemize}

\begin{sol}
	row1 = severe, row2 = medium, row3 = minor, col1 = A, col2 = B, col3 = C.

	Set the null and the alternative hypothesis.(+3)

	Expected frequency(+3)

	Test statistic and the observed value($5.02$)(+2)

p-value with chi-squared distribution, dof 4($0.28$), conclusion(+2)

	The procedure is the same as 10.4.2, so I omitted calculations.

	\qed
\end{sol}

\end{document}
	
