\subsection*{B. Properties of Lebesgue Measure} \hfill \\

\begin{enumerate}
	\item $\mathcal{L}$ is complement closed.
	\item $\mathcal{L}$ is $\sigma - \cap \cup$ closed.
	\item $\mathcal{L}$ is set minus closed.
	\item Countable additivity of $\lambda$.
	\item Continuity from below.
	\item Continuity from above.
	\item $\mathcal{L}$ contains Borel algebra on $\mathbb{R}^n$.
	\item $\lambda$ is complete.
	\item Theorem on approximation of $\mathcal{L}$.
	\item On $\mathcal{L}$, $\lambda ^{*} = \lambda_{*} = \lambda$.
	\item $\lambda(B) = \lambda^{*}(A) + \lambda_{*}(B\setminus A)$ if $A \subset B \in \mathcal{L}$.
	\item $A\in \mathcal{L}$ if and only if $\lambda^{*}(E) = \lambda^{*}(E\cap A)+\lambda^{*}(E\cap A^{c})$ for all $E \subset \mathbb{R}^n$.
\end{enumerate}

Problem 27. \\

If $A \subset \bigcup_{k=1}^{\infty}I_{k}$, then $\lambda^{*}(A) \leq \lambda^{*}(\bigcup_{k=1}^{\infty}I_{k}) \leq \sum_{k=1}^{\infty}\lambda^{*}(I_{k}) \leq \sum_{k=1}^{\infty}\lambda(I_{k})$. Therefore $\lambda^{*}(A) \leq \inf{\{\} }$.

On the contrary, assume $\lambda^{*}(A) < \inf{ \{ \} }$. There is open set $G$ containing $A$ and $\lambda(G) < \inf{ \{ \} }$. From problem 9, $G$ can be expressed as a countable union of nonoverlapping special rectangles. That is, $\lambda(G) = \sum_{k=1}^{\infty}\lambda(I_{k})$ which is contradiction.\\

Problem 29. \\

Use $\lambda(A \cup B) = \lambda(A \setminus B) + \lambda(B \setminus A) + \lambda(A\cap B)$.\\

Problem 30. \\

Let $G_{A}$, $G_{B}$ be open sets containing $A$, $B$ respectively.
$\lambda(G_A) + \lambda(G_B) = \lambda(G_A \cup G_B) + \lambda(G_A \cap G_B) \geq \lambda^{*}(A\cup B) + \lambda^{*}(A\cap B)$. Since $G_A$, $G_B$ are arbitrary, $\lambda^{*}(A) + \lambda^{*}(B) \geq \lambda^{*}(A\cup B) + \lambda^{*}(A \cap B)$.\\

Problem 31. \\

Clearly one point set is measure zero and closed. Let $\varepsilon >0$ be given. There exists open set $G_i$ containing $a_i$ such that $\lambda(G_i \setminus a_i) < {\varepsilon \over 2^i}$. Therefore $\lambda(A) <\varepsilon$ which means $\lambda(A) = 0$.\\

Problem 32. \\

Let $I_{k}$ be special 'cube' centered at origin, length of side $= k$. Then $a \times I_k \subset a \times \mathbb{R}^n$. Therefore we get $\lambda(a \times \mathbb{R}^n) = \lim_{k\rightarrow \infty}\lambda(a\times I_k) = 0$ by continuity from below.\\

Problem 34. \\

There is closed set of positive measure, no interior. \begin{equation}
	\left [ 0, 1 \right ] \setminus \bigcup_{k=1}^{\infty}\left (q_k - {\varepsilon \over 2^{k+1} }, q_k + {\varepsilon \over 2^{k+1} } \right )
\end{equation}

Problem 37. \\

For each positive integer $k$, choose open set $G_k supset E$ such that $\lambda(G_k) < \lambda^{*}(E) + {1 \over k}$. Put $A = \bigcap_{k=1}^{\infty}G_k$ which is measurable. Then $\lambda(A) < \lambda^{*}(E) + {1 \over k}$ for all positive integer $k$. Therefore $\lambda(A) \leq \lambda^{*}(E)$. Reverse inequality is trivial because $E \subset A$. We can conclude that there exists measurable hull of $E$ which has finite outer measure. \\

Problem 38. \\

First assume $A$ is measurable hull of $E$. $\lambda(A) = \lambda^{*}(E) < \infty$. But we already know that $\lambda(A) = \lambda^{*}(E) + \lambda_{*}(A \setminus E)$. Therefore $\lambda_{*}(A \setminus E) = 0$.

Conversely assume $\lambda_{*}(A\setminus E) = 0$. From 11st property of Lebesgue measure, we can get $A$ is measurable hull of $E$.\\

Problem 39. \\

Let $E_k = B(0, k) \setminus B(0, k-1)$ which is measurable partition of $\mathbb{R}^n$. Then $E \cap E_k$ has finite outer measure. By problem 37, there exists measurable hull $A_k$ of $E \cap E_k$. Put $A = \bigcup_{k=1}^{\infty}A_k$ and consider the compact set $K \subset A \setminus E$. But $\lambda(K \cap E_k ) \leq \lambda_{*}(A_k \setminus E\cap E_k) = 0$. By continuity from below, $\lambda(K) = 0$. So $\lambda_{*}(A\setminus E) = 0$.\\

Problem 40. \\

Let $A_{k}$ be measurable hull of $E_{k}$. $B_j = \bigcap_{i=j}^{\infty}A_{i}$ is also measurable hull of $E_{j}$ and $B_{i} \subset B_{i+1}$. Then $\lambda \left ( \bigcup_{k=1}^{\infty}B_{k}\right ) = \lim_{k\rightarrow \infty } \lambda \left ( B_{k} \right ) = \lim_{k\rightarrow \infty} \lambda^{*} \left ( E_k \right ) $.

Also we know that $\lambda^{*} \left ( \cup_{k=1}^{\infty} E_k \right ) \leq \lambda \left ( \bigcup_{k=1}^{\infty} B_{k} \right )$.
On the contrary, assume $\lambda^{*}\left ( \bigcup_{k=1}^{\infty}E_{k} \right ) < \lambda \left ( \bigcup_{k=1}^{\infty}B_{k}\right )$. There is open set $G$ containing $\bigcup_{k=1}^{\infty}E_k$ whose measure is strictly less than $\bigcup_{k=1}^{\infty}B_{k}$. But outer measure of $G$ is greater than $E_{k}$ for each positive integer $k$. Which is contradiction because $\lim_{k\rightarrow \infty}\lambda^{*}\left ( E_k \right ) = \lambda\left ( \bigcup_{k=1}^{\infty}B_{k} \right ) > \lambda \left (G \right ) = \lambda^{*}\left ( G \right )$. \\

Problem 41. \\

Put $B_{k} = \bigcup_{j=k}^{\infty}A_{j}$. Then $B_k \supset B_{k+1}$. $\lambda\left ( \limsup_{k\rightarrow \infty}A_{k} \right ) = \lambda \left ( \bigcap_{k=1}^{\infty}B_k \right ) = \lim_{k\rightarrow \infty}\lambda\left ( B_{k} \right )$. But $\lambda\left ( B_k \right ) \leq \sum_{j=k}^{\infty}\lambda \left ( A_{k} \right )$ which goes to $0$ as $k \rightarrow \infty$. Therefore $\lambda \left ( B_{k} \right ) \rightarrow 0$. \\

Problem 42. \\

Use hint. Or, let $h = \sum_{i=1}^{\infty}1_{A_i}$, then $h$ is measurable function. So inverse image of $\left [ 0, d \right ]$ under $h$ is measurable. Then we get

\begin{equation*}
	\sum_{i=1}^{\infty}\lambda \left ( A_i \right ) = \int_{h^{-1}\left ( \left [ 0, d \right ] \right )} h d\lambda \leq \int_{h^{-1}\left ( \left [ 0, d \right ] \right )}d d\lambda = d\lambda\left ( \bigcup_{k=1}^{\infty} A_{k} \right )
\end{equation*}

Problem 43. \\

$B_{k} = A_{k} \setminus \bigcup_{j=1}^{k-1}A_{j}$ when $k \geq 2$. Obviously $B_{1} = A_{1}$. Then $\lambda \left ( B_k \right ) = \lambda \left ( A_k \setminus A_1 \right ) = \lambda\left ( A_k \right )$. Therefore we get what we want.\\

Problem 44. \\

$\sum_{k=1}^{\infty} \left ( \lambda \left ( A_k \right ) - \lambda \left ( B_k \right ) \right ) = 0$. It leads $\lambda \left ( A_k \right ) = \lambda \left ( B_k \right )$. So $\lambda \left ( A_k \right ) = \lambda \left ( A_k \right ) - \lambda \left ( A_k \cap \left ( \bigcup _{j=1}^{k-1}A_j \right ) \right ) $ which leads conclusion.\\

Problem 45. \\

For each positive integer $k$, there are at most countable $A_i$'s such that $\lambda\left ( B(0, k) \cap A_i \right ) > 0$. Collect such $A_i$'s. Obviously such collection is countable. Then for some positive integer $k$, there are at most countable $A_i$'s in collection such that $\lambda \left ( B(0, k) \cap A_i \right ) > 0$. Actually, our collection is all of $\{ A_i : i \in \mathcal{I} \}$. Because if $A_i \cap B(0, k)$ has measure zero for all positive integer $k$, then $\lambda \left ( A_i \right ) = 0$ by continuity from below. This means $i \notin \mathcal{I} $. Therefore $\mathcal{I}$ is countable.\\

Problem 46. \\

Let $B_k = \bigcup_{j=k}^{\infty}A_j$. Then $B_k \supset B_{k+1}$. If $\lambda\left (B_1 \right ) < \infty$, continuity from above implies 

\begin{equation*}
	\lambda\left ( \limsup_{k\rightarrow \infty } A_k \right ) = \lim_{k \rightarrow \infty } \lambda \left ( B_k \right ) = \limsup_{k\rightarrow \infty} \lambda \left ( B_k \right ) \geq \limsup_{k\rightarrow \infty}\lambda \left ( A_k \right )
\end{equation*}
$\liminf$ case is similar to above.\\

Problem 47. \\

$\varepsilon \leq \limsup \lambda \left ( A_k \right ) \leq \lambda \left ( \limsup A_k \right )$. Therefore $\limsup A_k$ is not empty set because it has nonzero measure. \\

Problem 48. \\

Let $B_k$ be open ball of radius $k$ and centered at origin. Then $A \cap B_k \in \mathcal{L}_{0}$. There exist compact set $C_k$ such that $\lambda \left ( A\cap B_k \setminus C_k \right ) < {1 \over k} $. Set $K_k = \bigcup_{j=1}^{k}C_j \subset A\cap B_k$. Then $K_k \subset K_{k+1}$ and each $K_k$ is compact.
Then $\lambda \left ( A \cap B_k \cap \bigcap_{j=1}^{\infty}K_{j}^{c} \right ) \leq \lambda \left ( A \cap B_j \cap K_{j}^{c} \right ) < {1 \over j }$ for all $j > k$. So $\lambda \left ( A \cap B_k \cap \bigcap_{j=1}^{\infty} K_{j}^{c} \right ) = 0$. By continuity from below, we get what we want.\\



