\subsection*{section B. Borel Sets}\hfill \\

Borel algebra on $X$ is the smallest $\sigma$ algebra containing topology $\tau$ on $X$. Borel set denotes member of Borel algebra on $X$. It is not easy to construct exact Borel algebra on $\mathbb{R}^n$. It is useful to understand Borel algebra on $\mathbb{R}^{n}$ by smallest $\sigma$ algebra containing all the special rectangels. \\

Problem 12. \\

\begin{itemize}
    \item $G \in \tau \Rightarrow G = \bigcup_{k=1}^{\infty}I_k$ where $I_k$ is special rectangle. So $G \in \mathcal{M} \Rightarrow \tau \subset mathcal{M}$.
    \item $\tau \subset \mathcal{M} \Rightarrow \mathcal{B} \subset \mathcal{M}$.
    \item $I_k \in \mathcal{N}$. Since $I_k$ is closed, $I_k \in \mathbb{B}$. Therefore $\mathcal{N} \subset \mathcal{B}$.
    \item c implies $\mathcal{M} \subset \mathcal{B}$. \\
\end{itemize}

Problem 14. \\

By using approximation property of Lebesgue measure, $ \Rightarrow $ direction is clear.

On the contrary, assume there is $F_{\sigma}, G_{\delta}$ satisfying written properties.
Then $\lambda^{*}\left ( A \setminus F_{\sigma} \right ) \leq \lambda \left ( G_{\delta} \setminus F_{\sigma} \right ) = 0$.
Therefore $A \setminus F_{\sigma} \in \mathcal{L}$ because of completeness. Then $A = A\setminus F_{\sigma} \cup F_{\sigma} \in \mathcal{L}$. \\
