\section*{chapter 6}
\subsection*{section A. Nonnegative Functions}\hfill \\

Problem 1. "the vanishing property"\\

First, assume that $\int f d\mu = 0$. Since $f$ is measurable, there exists
an increasing seqeunce of nonnegative simple functions which converges to $f$. Let
denote them as $s_n$. Then $f^{-1}((0, \infty]) = \bigcup_{n=1}^{\infty}s_n ^{-1}((0, \infty])$, union of measure zero set.
Therefore, we get $\mu(f^{-1}((0, \infty])) = 0$.

Conversely, assume that $\int f d\mu > 0$. Then there exists nonnegative
simple function $s \leq f$ such that $\int s d\mu > 0$. Also we can write $s$ as
a linear combination of (measurable) characteristic functions, i.e. $\int s d \mu = \sum_{i=1}^N \alpha_i \mu(A_i) >0$.
So, $\alpha_i \mu(A_i) > 0$ for at least one integer $1 \leq i \leq N$. By Observing the fact that $A_i \subset f^{-1}((0, \infty])$,
we can conclude that $f^{-1}((0, \infty])$ has measure zero implies $\int f d \mu = 0$ by contrapositive.\\

Problem 2. "the finiteness property"\\

$E = \{x: f(x) = \infty \}$ is a measurable set since $f$ is measurable.
Suppose $\mu(E) > 0$. For any $M \in \mathbb{N}$, choose nonnegative simple measurable function $s_M \leq f$ such that $s_M(x) \geq M$ for $x \in E$.
Then $\int f d\mu \geq \int s_M d\mu \geq M \mu(E)$. Therefore, $\int f d\mu \geq M \mu(E)$ for all positive integer $M$.
It means that $\int f d\mu = \infty$. By taking contrapositive, we get what we want.\\

Problem 3. "compatibility to not finite nonnegative simple functions" \\

We are only interested in the case when $\alpha_k = \infty, \mu(A_k) > 0$.
Since $f$ is measurable, we can consider seqeunce of nonnegative finite simple function
$\{ t_n\}$ such that $t_n(x) \geq n$ for $x \in A_k$.
Therefore, $\int f d\mu \geq \int t_n d\mu \geq \mu(A_k)n$ for all positive integer $n$, which means $\int f d\mu = \infty$.
For other cases, it is easy to check the compatibility.\\

Problem 4. "scalar multiplication is still valid for $c = \infty$" \\

First, assume $\int f d\mu = 0$. Then $\mu(f^{-1}((0, \infty])) = 0$ by Problem 1.
It is obvious that $g^{-1}((0, \infty]) = f^{-1}((0, \infty])$ for $g = \infty f$.
So, $\int g d\mu = 0$. Therefore we get $\int \infty f d\mu = \int f d\mu$.

Second, assume $\int f d\mu > 0$. Then measure of $g ^ {-1} ((0, \infty ])$ is positive.
So, $\int \infty f d \mu = 0 \mu ( g ^ {-1} (0)) + \infty \mu(g^{-1}((0, \infty]))$ which is $\infty$.\\

Problem 5. "strict inequality for Fatou's lemma'\\

Consider this nonnegative finite simple function defined on real line:
\begin{equation*}
    s_n(x) = \begin{cases}
        n^2 & \text{if } x \in (0, {1 \over n}) \\
        0 & \text{otherwise}
    \end{cases}
\end{equation*}
$\lim s_n = 0$ and $\lim \int s_n d\mu = \infty$.\\

Problem 6.\\

Let $f_k = 1_{A_k}$ where $1_{A_k}$ is an indicator(characteristic) function.
By Fatou's lemma (ILLLI), $\int \liminf_{k \rightarrow \infty} 1_{A_k} d\mu \leq \liminf_{k \rightarrow \infty} \int 1_{A_k}d\mu = \liminf_{k \rightarrow \infty} \mu(A_k)$.
Now, showing $1_{\liminf A_k} \leq \liminf_{k\rightarrow \infty}1_{A_k}$ is left to us.
If $x \in \liminf A_k$ then $x \in \bigcap_{i\geq n} A_i$ for some positive integer $n$.
Then $1_{A_i}(x) = 1$ for all positive integer $i$ greater than $n$. So, $1_{\liminf A_k} \leq \inf_{i\geq n} 1_{A_i}$.
By letting $n \rightarrow \infty$, we get $1_{\liminf A_k} \leq \liminf_{k\rightarrow \infty}1_{A_k}$.
Therefore, $\mu(\liminf A_k) = \int 1_{\liminf A_k}d\mu \leq \int \liminf_{k \rightarrow \infty} 1_{A_k} d\mu \leq \liminf_{k \rightarrow \infty} \int 1_{A_k}d\mu = \liminf_{k \rightarrow \infty} \mu(A_k)$.\\
