\subsection*{section C} \hfill \\

Problem 16. \\

Clearly, $ \lvert f \rvert = 0$ almost everywhere, and $\lvert f \rvert$ is measurable.
Consider nonnegative finite simple function $s \leq \lvert f \rvert$. Then $s=0$ almost everywhere
and $s = \sum_{i=1}^N \alpha_i 1_{A_i}$, where $A_i$ is null set if $\alpha_i \ne 0$.
Therefore, $\int s d\mu = 0$, which implies $\int \lvert f \rvert d\mu = 0$.
So, $f \in \mathcal{L}^1(\mu)$, $\left | \int f d \mu \right | =0 \Rightarrow \int f d \mu = 0$.\\

Problem 17. \\

Note that $f \sim g \Leftrightarrow f = g\text{ a.e.}$ is an equivalence relation.
So, $g = h \text{ a.e}$.
Let g be measurable and $E_t = \left [ -\infty, t \right]$, $N = \{ x : g(x) \ne h(x)\}$.
Then $h^{-1}(E_t) \setminus g^{-1}(E_t) \subset N$ and $g^{-1}(E_t) \setminus h^{-1}(E_t) \subset N$.
Since $\mu$ is complete, they are all null sets. So $h^{-1}(E_t) \cup g^{-1}(E_t)$ is measurable.
Therefore, $h^{-1}(E_t)$ is also measurable, which implies measurability of $h$.\\
