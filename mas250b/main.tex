\documentclass{article}
\usepackage{amsmath, amssymb, enumitem}

\begin{document}
\section*{Homework 4}
\begin{enumerate}
	\item[40.]
		It is maximized when $(p_1, p_2, p_3) = (1/2, 0, 1.2)$ and minimized when $(0, 1, 0)$.

	\item[42.]
		Use $V(X) = EX^2 - (EX)^2$.
		Equality holds iff $V(X) = 0$, i.e. $P(X = EX) = 1$.

	\item[44.]
		\begin{enumerate}[label = (\alph*)]
			\item Since we have joint density,
				\[
					f_{X_1}(x) = \int_{0}^{1-x}3(x+y)dy = \frac{3}{2} - \frac{3}{2}x^2
				\]
				where $0<x<1$.
				$f_{X_2}$ can be get similarly.

			\item Mean is
				\[
					\int_0^1 xf_{X_1}(x) dx = 3/8,
				\]
				and the second moment is
				\[
					\int_0^1 x^2 f_{X_1}(x) dx = 1/5.
				\]
		\end{enumerate}

	\item[49.]
		Follow the hint.
		\[
			0\le V\left (\frac{X}{\sigma_X} + \frac{Y}{\sigma_Y} \right ) = 2+2corr(X,Y).
		\]
		When $corr(X,Y) = -1$, then
		\[
			V\left( \frac{X}{\sigma_X} + \frac{Y}{\sigma_Y} \right) = 0.
		\]
		This implies $X/\sigma_X + Y/\sigma_Y$ is constant.
		So linearity follows.
		We can prove the other part similarly.

	\item[50.]
		Follow the hint.
		\[
			cov(N_1, N_2) = \sum_i \sum_j cov(X_i, Y_j) = cov(X_i, Y_i) = -np_1 p_2
		\]
		where the second inequality is due to the independence.

	\item[52.]
		$= E(X_1^2 - X_2^2) - E(X_1 - X_2)E(X_1 + X_2) = E(X_1^2 - X_2^2) = 0$
		because they have same distribution.

	\item[54.]
		The moment generating function of $X$ is
		\[
			M_X(t) = \int_0^1 \exp(tx) dx = \frac{\exp(t)-1}{t} = \sum_{n\ge 0} \frac{t^n}{(n+1)!}.
		\]
		Thus $EX^n$ is equal to $n$-th derivative of $M_X(t)$ at $t = 0$, which is $1/(n+1)$.
		This result agrees with
		\[
			EX^n = \int_0^1 x^n dx = \frac{1}{n+1}.
		\]

	\item[57.]
		Note that the distribution function determines the distribution of given random variable.
		That is, if two random variable have same distribution function, then they have same distribution.
		So they have same moments.
		This does not mean that two random variables are same.

		Observe that
		\[
			P(X \le x) = P(Y \le \frac{x-a}{b}) = P(a+bY \le x).
		\]
		Thus $X$ and $a+bY$ have same distribution.
		So $EX = E(a+bY) = a+bEY$, and $VX = b^2 VY$.
\end{enumerate}

\section*{Homework 5}
\begin{enumerate}
	\item[1.] Let $X$ be the number of components in working condition.
		Then $X$ follows binomial$(4, .6)$ distribution.
		Thus
		\[
			P(X \ge 2)
			= 1-P(X = 0) - P(X = 1)
			= 1-.4^4 - {4 \choose 1}.6^1 .4^3.
		\]
		
	\item[5.] Probability of working condition for $2$-engine plane is $1-(1-p)^2$.
		Corresponding probability of $4$-engine plane is $1-(1-p)^4 - {4\choose 1}(1-p)^3p$.
		Now it is sufficient to solve that
		\[
			(1-p)^4 + {4\choose 1}p(1-p)^3 \le (1-p)^2.
		\]
		The answer is $p \ge 2/3$.

	\item[9.] $X$ follows binomial$(n, p)$ distribution.
		\[
			\begin{split}
				E e^{tX}
				&= \sum_{k=0}^n e^{tk} {n\choose k} p^k (1-p)^{n-k} \\
				&= \sum {n \choose k} (pe^t)^k (1-p)^{n-k} \\
				&= (1-p+pe^t)^n \sum {n\choose k}\left( \frac{pe^t}{1-p+pe^t} \right)^k\left( \frac{1-p}{1-p+pe^t} \right)^{n-k} \\
				&= (1-p+pe^t)^n.
			\end{split}
		\]
		First derivative at $t = 0$ is $np$, and the second derivative at $t = 0$ is $n(n-1)p^2 +np$.
		Thus we get the answer.

	\item[11.] Let $X$ be the number of winnings.
		Then $X$ follows binomial$(50, .01)$ distribution.
		Using this, we can exact probabilities.

		But one may want another method, because the calculations are messy.
		Actually since $n$ is large and $p$ is small, we can regard $X$ as Poisson $50 \cdot .01 = 1/2$ random variable approximately.
		Using this approximation,
		\[
			P(X \ge 1) = 1-P(X = 0) = 1-\exp(-1/2),
		\]
		\[
			P(X = 1) = \exp(-1/2)1/2,
		\]
		\[
			P(X \ge 2) = 1- P(X = 0) - P(X = 1) = 1- \exp(-1/2) - 1/2 \exp(-1/2).
		\]

	\item[12.] Let $X$ be the number of colds.
		Let $Y$ be the indicator function of the event: 'the drug is effective'.

		What we want is $P(Y = 1 \lvert X = 0)$.
		By using Bayes' formula,
		\[
			\begin{split}
				P(Y = 1 \lvert X = 0) 
				&= \frac{P(X = 0, Y = 1)}{P(X = 0, Y = 1) + P(X = 0, Y = 0)}\\
				&= \frac{0.75 \exp(-2)}{0.75 \exp(-2) + 0.25 \exp(-3)}.
			\end{split}
		\]

	\item[19.]
		\begin{enumerate}[label = (\alph*)]
			\item By direct calulation,
				\[
					\frac{P(X=i)}{P(X = i-1)} = \frac{(n-i+1)(k-i+1)}{i(m-k+i)}.
				\]

			\item This can be done by the recurrence relation above.
		\end{enumerate}

	\item[20.]
		\begin{enumerate}[label = (\alph*)]
			\item $(1-p)^{k-1}p.$

			\item Let $S = \sum_{k\ge 1} k(1-p)^{k-1}p$.
				Then 
				\[
					S - (1-p)S = pS = \frac{p}{1-(1-p)}= 1.
				\]
				So $S = 1/p$.

			\item There are $r-1$ success until $k-1$.
			Thus 
			\[
				P(Y = k) = {k-1 \choose r-1} p^r (1-p)^{k-r}.
			\]

			\item $Y = \sum_{i=1}^r Y_i$, where $Y_i$ is iid geometric r.v. considered in (a).
				Thus $EY = \sum_{i=1}^rEY_i = r \cdot 1/p$.
				Note that direct calculation is also possible.
		\end{enumerate}

	\item[21.] First,
		\[
			P(a+(b-a)U \le x) = P\left( U \le \frac{x-a}{b-a} \right).
		\]
		For  $\frac{x-a}{b-a} \in (0, 1)$, the above is equal to $\frac{x-a}{b-a}$.
		But $\frac{x-a}{b-a} \in (0, 1)$ is equivalent to $a < x < b$.
		Also observe that $x \ge b$ implies the above is equal to $1$, and $x \le a$ leads the above $0$.
		Thus distribution functions of $a+(b-a)U$ and $Unif(a, b)$ are the same.
\end{enumerate}
\end{document}
