\documentclass{article}
\usepackage{amsmath, amssymb, enumitem}
\usepackage{xcolor}

\begin{document}
\section*{Homework 4}
\begin{enumerate}
	\item[40.]
		It is maximized when $(p_1, p_2, p_3) = (1/2, 0, 1.2)$ and minimized when $(0, 1, 0)$.

	\item[42.]
		Use $V(X) = EX^2 - (EX)^2$.
		Equality holds iff $V(X) = 0$, i.e. $P(X = EX) = 1$.

	\item[44.]
		\begin{enumerate}[label = (\alph*)]
			\item Since we have joint density,
				\[
					f_{X_1}(x) = \int_{0}^{1-x}3(x+y)dy = \frac{3}{2} - \frac{3}{2}x^2
				\]
				where $0<x<1$.
				$f_{X_2}$ can be get similarly.

			\item Mean is
				\[
					\int_0^1 xf_{X_1}(x) dx = 3/8,
				\]
				and the second moment is
				\[
					\int_0^1 x^2 f_{X_1}(x) dx = 1/5.
				\]
		\end{enumerate}

	\item[49.]
		Follow the hint.
		\[
			0\le V\left (\frac{X}{\sigma_X} + \frac{Y}{\sigma_Y} \right ) = 2+2corr(X,Y).
		\]
		When $corr(X,Y) = -1$, then
		\[
			V\left( \frac{X}{\sigma_X} + \frac{Y}{\sigma_Y} \right) = 0.
		\]
		This implies $X/\sigma_X + Y/\sigma_Y$ is constant.
		So linearity follows.
		We can prove the other part similarly.

	\item[50.]
		Follow the hint.
		\[
			cov(N_1, N_2) = \sum_i \sum_j cov(X_i, Y_j) = cov(X_i, Y_i) = -np_1 p_2
		\]
		where the second inequality is due to the independence.

	\item[52.]
		$= E(X_1^2 - X_2^2) - E(X_1 - X_2)E(X_1 + X_2) = E(X_1^2 - X_2^2) = 0$
		because they have same distribution.

	\item[54.]
		The moment generating function of $X$ is
		\[
			M_X(t) = \int_0^1 \exp(tx) dx = \frac{\exp(t)-1}{t} = \sum_{n\ge 0} \frac{t^n}{(n+1)!}.
		\]
		Thus $EX^n$ is equal to $n$-th derivative of $M_X(t)$ at $t = 0$, which is $1/(n+1)$.
		This result agrees with
		\[
			EX^n = \int_0^1 x^n dx = \frac{1}{n+1}.
		\]

	\item[57.]
		Note that the distribution function determines the distribution of given random variable.
		That is, if two random variable have same distribution function, then they have same distribution.
		So they have same moments.
		This does not mean that two random variables are same.

		Observe that
		\[
			P(X \le x) = P(Y \le \frac{x-a}{b}) = P(a+bY \le x).
		\]
		Thus $X$ and $a+bY$ have same distribution.
		So $EX = E(a+bY) = a+bEY$, and $VX = b^2 VY$.
\end{enumerate}

\section*{Homework 5}
\begin{enumerate}
	\item[1.] Let $X$ be the number of components in working condition.
		Then $X$ follows binomial$(4, .6)$ distribution.
		Thus
		\[
			P(X \ge 2)
			= 1-P(X = 0) - P(X = 1)
			= 1-.4^4 - {4 \choose 1}.6^1 .4^3.
		\]
		
	\item[5.] Probability of working condition for $2$-engine plane is $1-(1-p)^2$.
		Corresponding probability of $4$-engine plane is $1-(1-p)^4 - {4\choose 1}(1-p)^3p$.
		Now it is sufficient to solve that
		\[
			(1-p)^4 + {4\choose 1}p(1-p)^3 \le (1-p)^2.
		\]
		The answer is $p \ge 2/3$.

	\item[9.] $X$ follows binomial$(n, p)$ distribution.
		\[
			\begin{split}
				E e^{tX}
				&= \sum_{k=0}^n e^{tk} {n\choose k} p^k (1-p)^{n-k} \\
				&= \sum {n \choose k} (pe^t)^k (1-p)^{n-k} \\
				&= (1-p+pe^t)^n \sum {n\choose k}\left( \frac{pe^t}{1-p+pe^t} \right)^k\left( \frac{1-p}{1-p+pe^t} \right)^{n-k} \\
				&= (1-p+pe^t)^n.
			\end{split}
		\]
		First derivative at $t = 0$ is $np$, and the second derivative at $t = 0$ is $n(n-1)p^2 +np$.
		Thus we get the answer.

	\item[11.] Let $X$ be the number of winnings.
		Then $X$ follows binomial$(50, .01)$ distribution.
		Using this, we can exact probabilities.

		But one may want another method, because the calculations are messy.
		Actually since $n$ is large and $p$ is small, we can regard $X$ as Poisson $50 \cdot .01 = 1/2$ random variable approximately.
		Using this approximation,
		\[
			P(X \ge 1) = 1-P(X = 0) = 1-\exp(-1/2),
		\]
		\[
			P(X = 1) = \exp(-1/2)1/2,
		\]
		\[
			P(X \ge 2) = 1- P(X = 0) - P(X = 1) = 1- \exp(-1/2) - 1/2 \exp(-1/2).
		\]

	\item[12.] Let $X$ be the number of colds.
		Let $Y$ be the indicator function of the event: 'the drug is effective'.

		What we want is $P(Y = 1 \lvert X = 0)$.
		By using Bayes' formula,
		\[
			\begin{split}
				P(Y = 1 \lvert X = 0) 
				&= \frac{P(X = 0, Y = 1)}{P(X = 0, Y = 1) + P(X = 0, Y = 0)}\\
				&= \frac{0.75 \exp(-2)}{0.75 \exp(-2) + 0.25 \exp(-3)}.
			\end{split}
		\]

	\item[19.]
		\begin{enumerate}[label = (\alph*)]
			\item By direct calulation,
				\[
					\frac{P(X=i)}{P(X = i-1)} = \frac{(n-i+1)(k-i+1)}{i(m-k+i)}.
				\]

			\item This can be done by the recurrence relation above.
		\end{enumerate}

	\item[20.]
		\begin{enumerate}[label = (\alph*)]
			\item $(1-p)^{k-1}p.$

			\item Let $S = \sum_{k\ge 1} k(1-p)^{k-1}p$.
				Then 
				\[
					S - (1-p)S = pS = \frac{p}{1-(1-p)}= 1.
				\]
				So $S = 1/p$.

			\item There are $r-1$ success until $k-1$.
			Thus 
			\[
				P(Y = k) = {k-1 \choose r-1} p^r (1-p)^{k-r}.
			\]

			\item $Y = \sum_{i=1}^r Y_i$, where $Y_i$ is iid geometric r.v. considered in (a).
				Thus $EY = \sum_{i=1}^rEY_i = r \cdot 1/p$.
				Note that direct calculation is also possible.
		\end{enumerate}

	\item[21.] First,
		\[
			P(a+(b-a)U \le x) = P\left( U \le \frac{x-a}{b-a} \right).
		\]
		For  $\frac{x-a}{b-a} \in (0, 1)$, the above is equal to $\frac{x-a}{b-a}$.
		But $\frac{x-a}{b-a} \in (0, 1)$ is equivalent to $a < x < b$.
		Also observe that $x \ge b$ implies the above is equal to $1$, and $x \le a$ leads the above $0$.
		Thus distribution functions of $a+(b-a)U$ and $Unif(a, b)$ are the same.
\end{enumerate}

\section*{Homework 6}
\begin{enumerate}
	\item[31.]
		\begin{enumerate}[label = (\alph*)]
			\item Mean is $250,000$.
				By using $P(Z \ge 70,000/\sigma) = 0.25$, we can find $\sigma$ from the table.

			\item By using the value of $\sigma$ found above, we can calculate
				$P(10,000/ \sigma \le Z \le 50,000/\sigma)$.
		\end{enumerate}

	\item[32.]
		Let $X, Y$ be scores of econ, stat exams respectively.
		Then $P(X \le 70) = P(Z \le .5)$, and $P(Y \le 62) = P(Z \le .7)$.
		Thus
		\begin{enumerate}[label = (\alph*)]
			\item stat. exam
			\item $P(Z \le .5)$
			\item $P(Z \le .7)$.
		\end{enumerate}

	\item[37.]
		Let $X$ follow exponential distribution with mean $1$.
		Then its pdf is $\exp(-x)1_{\left( 0, \infty \right)}(x)$.
		\begin{enumerate}[label = (\alph*)]
			\item $P(X \ge 2) = \int_2^\infty\exp(-x)dx = \exp(-2).$
			\item $P(X \ge 3 \lvert X>2) = P(X \ge 1) = \exp(-1)$ by memoryless property.
		\end{enumerate}

	\item[39.]
		Let $X$ be an exponential r.v. with mean $20$.
		Then $P(X \ge 30 \lvert X \ge 10) = P(X \ge 20) = \exp(-20/20) = \exp(-1)$.

		Let $Y$ be an uniform r.v. supported on $(0, 40)$.
		Then $P(Y \ge 30 \lvert Y \ge 10) = \int_{30}^{40}1/40 dx / \int_{10}^{40} 1/40 dx = 1/3$.

	\item[43.]
		You can use the table for chi-squre distribution, or using the relationship between chi-square distribution and the gamma distribution.
		Note that $\chi_6^2$ r.v. has a pdf $t^2 \exp(-t/2)/16$.
		\begin{enumerate}[label = (\alph*)]
			\item What we want is $\int_0^6 t^2 \exp(-t/2) /16 dt$.
				We can calculate this by hand(integration by parts) or using calculator.

			\item What we want is $P(3 \le X \le 9) = \int_{3/2}^{9/2}x^2 \exp(-x)/2 dx$.
				We can calculate this value by hand or using calculator.
		\end{enumerate}

	\item[44.]
		Since two chi-squre r.vs are independent, their sum is $\chi_9^2$.
		Thus
		\[
			P(X + Y > 10) = \int_{10}^\infty\frac{t^{7/2}\exp(-t/2)}{\Gamma(9/2)2^{9/2}}dt.
		\]
		We can calculate this value by hand(integration by parts, and standard normal distribution) or using calulator.

	\item[45.]
		The density of gamma$(1/2, 1)$ variable is $x^{-1/2}\exp(-x)/\Gamma(1/2)$.
		Thus
		\[
			\begin{split}
				\Gamma(1/2) 
				&= \int_0^\infty \exp(-x) x^{-1/2} dx \\
				&= \int_0^\infty \exp(-y^2/2)\sqrt{2} dy\\
				&= 2\sqrt{\pi}\int_0^\infty\frac{\exp(-y^2/2)}{\sqrt{2\pi}}dy = \sqrt{\pi}.
			\end{split}
		\]
		Note that we used the density of standard normal variable.

	\item[48.]
		Let $Z$ be the standard normal variable.
		Then
		\[
			P(X \le x ) = P\left( Z \le \frac{x-a}{b} \right) = P(a+bZ \le x).
		\]
		So $X$ and $a +bZ$ are equal 'in distribution'.
		Since $a +bZ$ follows $n(a, b^2)$, we can say that $X$ also follows $n(a, b^2)$.
\end{enumerate}

\begin{enumerate}
	\item [3.]
		First, $A \cap B \subset B$.
		So
		\[
			P(A \cap B) \le P(B) = \frac{1}{3}.
		\]\textcolor{red}{(+5)}

		Second,
		\[
			\begin{split}
				P(A \cap B)
				&= 1- P\left( (A\cap B)^c \right) \\
				&= 1- P(A^c) -P(B^c) + P\left( A^c \cap B^c \right) \\
				&\ge 1-{1 \over 4} -{2\over 3} = {1 \over 12}.
			\end{split}
		\]\textcolor{red}{(+5)}

	\item [9.]
		By the definition of expectation,
		\[
			E[Xf(X)] = \frac{1}{\sqrt{2\pi}\sigma}\int xf(x) \exp\left (-\frac{x^2}{2\sigma^2} \right )dx.
		\]\textcolor{red}{(+2)}

		Using integration by parts, the above is equal to
		\[
			\frac{-\sigma}{\sqrt{2\pi}}f(x) \exp\left( \frac{-x^2}{2\sigma^2} \right)
			\lvert _{-\infty}^{\infty} + 
			\frac{\sigma^2}{\sqrt{2\pi}\sigma}\int f'(x) \exp\left( -\frac{x^2}{2\sigma^2} \right)dx.
		\]\textcolor{red}{(+3)}

		The first term of the above is equal to $0$, because $f$ is bounded. \textcolor{red}{(+2)}

		The last term is equal to $\sigma^2 E\left[ f'(X) \right]$. \textcolor{red}{(+3)}

	\item [11.]
		Let $X$ be a normal random variable with zero mean, $\sigma^2$ variance.
		Then
		\[
			E\left[ X^3 \right] = \int x^3 \frac{1}{\sqrt{2\pi}\sigma}\exp\left( -\frac{x^2}{2\sigma^2} \right)dx = 0
		\]
		since the integrand is an odd function.\textcolor{red}{(+2)}

		Now, notice
		\[
			E\left[ (X_1 + \cdots X_{10} )^3 \right]
			= \sum_{1 \le i \le 10} \sum_{1 \le j \le 10} \sum_{1 \le k \le 10}E\left[ X_i X_j X_k \right].
		\]\textcolor{red}{(+2)}

		If $i = j = k$, then $EX_iX_jX_k = EX_i^3 = 0$.

		If $i = j \ne k$, then $= EX_i^2 EX_k = 0$ by the independence.\textcolor{red}{(+2)}

		If $i, j, k$ are all distinct, then $= EX_i EX_j EX_k = 0$ by the independence. \textcolor{red}{(+2)}

		Therefore it is equal to $0$. \textcolor{red}{(+2)}

		\emph{
		It can be solved by using the mgf method.
		\begin{enumerate}
			\item If your mgf is incorrect, then \textcolor{red}{(-2)}.
			\item If your differentiation procedure is wrong, then \textcolor{red}{(-2)}.
			\item If you omit the explanation about 'mgf of sums are product of mgf', then \textcolor{red}{(-2)}.
	\end{enumerate}}
\end{enumerate}

\end{document}
