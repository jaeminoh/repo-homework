\begin{enumerate}
	\item [3.]
		First, $A \cap B \subset B$.
		So
		\[
			P(A \cap B) \le P(B) = \frac{1}{3}.
		\]\textcolor{red}{(+5)}

		Second,
		\[
			\begin{split}
				P(A \cap B)
				&= 1- P\left( (A\cap B)^c \right) \\
				&= 1- P(A^c) -P(B^c) + P\left( A^c \cap B^c \right) \\
				&\ge 1-{1 \over 4} -{2\over 3} = {1 \over 12}.
			\end{split}
		\]\textcolor{red}{(+5)}

	\item [9.]
		By the definition of expectation,
		\[
			E[Xf(X)] = \frac{1}{\sqrt{2\pi}\sigma}\int xf(x) \exp\left (-\frac{x^2}{2\sigma^2} \right )dx.
		\]\textcolor{red}{(+2)}

		Using integration by parts, the above is equal to
		\[
			\frac{-\sigma}{\sqrt{2\pi}}f(x) \exp\left( \frac{-x^2}{2\sigma^2} \right)
			\lvert _{-\infty}^{\infty} + 
			\frac{\sigma^2}{\sqrt{2\pi}\sigma}\int f'(x) \exp\left( -\frac{x^2}{2\sigma^2} \right)dx.
		\]\textcolor{red}{(+3)}

		The first term of the above is equal to $0$, because $f$ is bounded. \textcolor{red}{(+2)}

		The last term is equal to $\sigma^2 E\left[ f'(X) \right]$. \textcolor{red}{(+3)}

	\item [11.]
		Let $X$ be a normal random variable with zero mean, $\sigma^2$ variance.
		Then
		\[
			E\left[ X^3 \right] = \int x^3 \frac{1}{\sqrt{2\pi}\sigma}\exp\left( -\frac{x^2}{2\sigma^2} \right)dx = 0
		\]
		since the integrand is an odd function.\textcolor{red}{(+2)}

		Now, notice
		\[
			E\left[ (X_1 + \cdots X_{10} )^3 \right]
			= \sum_{1 \le i \le 10} \sum_{1 \le j \le 10} \sum_{1 \le k \le 10}E\left[ X_i X_j X_k \right].
		\]\textcolor{red}{(+2)}

		If $i = j = k$, then $EX_iX_jX_k = EX_i^3 = 0$.

		If $i = j \ne k$, then $= EX_i^2 EX_k = 0$ by the independence.\textcolor{red}{(+2)}

		If $i, j, k$ are all distinct, then $= EX_i EX_j EX_k = 0$ by the independence. \textcolor{red}{(+2)}

		Therefore it is equal to $0$. \textcolor{red}{(+2)}

		\emph{
		It can be solved by using the mgf method.
		\begin{enumerate}
			\item If your mgf is incorrect, then \textcolor{red}{(-2)}.
			\item If your differentiation procedure is wrong, then \textcolor{red}{(-2)}.
			\item If you omit the explanation about 'mgf of sums are product of mgf', then \textcolor{red}{(-2)}.
	\end{enumerate}}
\end{enumerate}
