\begin{problem}[10.1] \hfill
	\begin{enumerate}[label = (\alph*)]
		\item Let $\cos z = g(z)$. $g'(z) = -\sin z$ hence $g'(\pi/2) = -1 \ne 0$. Note that $g(\pi/2) = 0$.
			So, by theorem 5.2.2, there are $\delta, \varepsilon > 0$ such that each $q \in D(0, \varepsilon)$ has unique inverse image under $g$, and the inverse image of $q$ lies in $D(\pi/2, \delta)$.
			It is well known that $f: q \mapsto g^{-1}(q)$ on $D(0, \varepsilon)$ is holomorphic.
			
			Therefore, we have the function element $(f, U)$.
			Uniqueness(up to $\varepsilon$) follows from the uniqueness of inverse image of $g$ on $D(0, \varepsilon)$.

		\item Let $\alpha$ be any complex number.
			$\cos w = \alpha$ can be rewritten as $t^2 - 2\alpha t + 1 = 0$ for $t = e^{iw}$.
			The former equation has order 2 solution when $\alpha = \pm 1$, and otherwise, has simple two solutions.
			Thus, when $\alpha \ne \pm 1$, by choosing one of two solutions, we can apply theorem 5.2.2 again.
			So we can find function element of $\arccos$ $(f, U)$ where $U$ is a disc centered at $\alpha$, whose preimage under $g(z) = \cos z$ contains one of two solutions as described before.

			Now, let $\Delta \subset \mathbb{C} \setminus \left\{ -1, 1 \right\}$ be a disc, where $g^{-1}$ is well defined and holomorphic.
			Then $(f, \Delta)$ is a function element of $\arccos$.
			Let $D = g^{-1}(\Delta)$. Then $g$ is a conformal mapping of $D$ onto $\Delta$.

			When $\Delta \cap \Delta' \ne \emptyset$, we can find corresponding $D, D'$ which intersects. 
			Then, by letting $f$ be the inverse of $g|_{D}$, we can get the function element $(f, \Delta)$.
			Since $g$ has unique inverse on $D \cap D'$, $f = f'$ on $\Delta \cap \Delta'$.
			Thus $(f, \Delta), (f', \Delta')$ are direct analytic continuation.
			This process may be continued.

			Let $\gamma$ be a path from the origin to $\alpha \in \mathbb{C} \setminus \left\{ -1, 1 \right\}$.
			From the origin, we can apply the above procedure along $\gamma$.
			Then, by using compactness of the image of $\gamma$, we can cover the image by finite chain of $\Delta_i$ such that $(f_i, \Delta_i)$ is a direct analytic continuation of $(f_{i-1}, \Delta_{i-1})$.

			It says that $(f, U)$ from (a) admits unrestricted continuation in $\mathbb{C} \setminus\left\{ -1, 1 \right\}$.

		\item If $z_0 = \pm 1$, then $ \sin h(z_0)  = 0$.
			By chain rule, $-\sin h(z) h'(z) = 1$. Putting $z=z_0$ leads contradiction.
			So $z_0 \ne \pm 1$.

			Now, let $(f, U)$ be that of (a) and note that for given $\Delta \subset \mathbb{C} \setminus\left\{ -1, 1 \right\}$ which is a disc centered at $\alpha$, there are exactly two function elements by solving the equation $\cos w = \alpha$.	
			Let $\gamma_1, \gamma_2, \gamma_3$ be curves where the index of $\gamma_1 - \gamma_2$ at $1$ is $\pm 1$ but $0$ at $-1$, and the inndex of $\gamma_1 - \gamma_3$ at $-1$ is $\mp 1$ but $0$ at $1$.

			Then $(h, \Delta)$ can be achieved by analytic continuation along one of $\gamma_i$'s.
			Because, if not, $(f, U)$ defines global $\arccos$ on $\mathbb{C} \setminus \left\{ -1, 1 \right\}$ which is impossible.

			Impossibility follows from this observation: Let $\delta(t) = 1+\varepsilon e^{2\pi i t}$. Analytic continuation of $(f, U)$ along $\delta$ leads another function element defined on $U$ which is a disc centered at the origin.
			In fact, this observation leads the conclusion: given $(h, \Delta)$ is a member of equivalence class determined by $(f, U)$.
	\end{enumerate}
\qed
\end{problem}

\begin{problem}[10.2] \hfill
	\begin{enumerate}[label = (\alph*)]
		\item Let $u = s+t, v = t/s$. Then the integral must be:
			\[
				\int_0^\infty \int_0^\infty \frac{v^{z-1}}{1+v}e^{-u}dudv = \int_0^\infty \frac{v^{z-1}}{v+1}dv
			\]
			Since $0<1-z<1$, by calculating residue (similar to \#4 (a) of hw4), we can get $ = \pi / \sin \pi(1-z) = \pi / \sin \pi z$.

			From holomorphy of $\Gamma(z) \Gamma(1-z), \pi/\sin\pi z$ on $\mathbb{C} \setminus \mathbb{Z}$, they are same by the identity theorem.

		\item Note that $\Gamma(z) = \int_0^1 e^{-t}t^{z-1} dt + \int_1^\infty e^{-t}t^{z-1}dt = S_1 + S_2$.
			Then $|S_2| \leq \int_1^\infty e^{-t}t^{s-1} dt$ where $s = Re(z)$.
			When $s \geq 1/2$, take $s \leq n \leq s+1$.
			For such $n$, $|S_2| \leq \int_1^\infty e^{-t}t^n dt = \Gamma(n+1) = n! \leq n^n = e^{n\log n}
			\leq e^{(s+1) \log(s+1)}$.
			Since $| \sin \pi z | \leq e^{|z|}$, $\left | \frac{\sin \pi z}{\pi} \Gamma(z) \right | \leq e^{C|z| \log |z|}$.

			$|S_1| \leq |\int_0^1 \sum_{n=0}^\infty t^{n+s-1}(-1)^n/n! dt| = |\sum_{n=0}^\infty \frac{(-1)^n}{n!(n+s)}|$.
			But, the last term is bounded by constant if $s\geq 1/2$.

			Thus the result holds for $Re(z) = s \geq 1/2$.

		\item First, (a) says that $1/\Gamma$ is entire and has simple zeros at nonnegative integers.
			Then (b) says that the order of entire function $1/\Gamma$ is $1$.
			Thus the Hadamard factorization theorem implies:
			\[
				\frac{1}{\Gamma(z)} = e^{Az + B} z \Pi_{n=1}^\infty \left( 1+\frac{z}{n} \right)e^{-z \over n}
			\]
			Note that $B = 0$ by considering $z \Gamma(z) = \Gamma(z+1) \rightarrow 1$ as $z \rightarrow 0$.
			Next, by putting $z = 1$,
			\[
				\begin{split}
					e^{-A} 
					& = \Pi_{n=1}^{\infty}\left( 1+1/n \right)e^{-1 \over n} \\
					& = \exp(\sum_{n=1}^\infty \left( \log (1+ {1\over n} ) - {1 \over n} \right) \\
					& = \lim_{N\rightarrow \infty} \exp\left( \sum_{n=1}^N (\log(1+1/n) - 1/n \right) \\
					& = \lim_{N\rightarrow \infty} \exp\left( -\sum_{n=1}^N 1/n + \log N +\log(1+1/N) \right) \\
					& = e^{-\gamma}
				\end{split}
			\]

			\qed
	\end{enumerate}
	
\end{problem}

\begin{problem}[10.3] \hfill

	Every element of $\Gamma$ can be expressed as finite product of $\mu, \omega, \mu^{-1}, \omega^{-1}$.
	So, we'll use the induction on the length of $f \in \Gamma$.

	First, for length 1 $f$, the assertion trivially holds.
	Let length of $f$ be $n$. Then,
	\[ 
		\mu \circ f (z) = \frac{az + b}{(2a+c)z + 2b+d}
	\]
	where $a, d$ are odds and $b, c$ are evens.
	Thus $\mu \circ f$ satisfies the assertion.
	Also, 
	\[
		\omega \circ f (z) = \frac{(a+c)z + b+d}{cz + d}
	\]
	so $\omega \circ f$ satisfies the assertion.

	Similarly, for the inverses of $\mu, \omega$, we can check the assertion.
	Therefore, the assertion holds by induction.

	\qed
	
\end{problem}

\begin{problem}[10.4] \hfill
	\begin{enumerate}[label = (\alph*)]
		\item Since $f$ is doubly periodic, it is sufficient to show that the residue of $f$ at the origin is zero.
			Let
			\begin{align*}
				\gamma_1(t) & = \frac{1}{2} it \\
				\gamma_2(t) & = -t +i\frac{1}{2} \\
				\gamma_3(t) & = -\frac{1}{2} -it \\
				\gamma_4(t) & = t -i\frac{1}{2}
			\end{align*}
			where $-1/2 \leq t \leq 1/2$.
			Then by adjoining the above paths, we get the curve $\gamma$ whose image is the square centered at the origin.

			Now, integrate $f$ along $\gamma$. Then
			\[
				\int_\gamma f = \sum_{i=1}^4 \int_{\gamma_{i}} f
			\]
			But we can easily check that integral of $f$ along $\gamma_i$ and $\gamma_{i+2}$ are cancelled by its double periodicity($i = 1, 2$).
			Thus $\int_\gamma f = 0$.
			Therefore, by the residue theorem, $Res_0(f) = 0$.
			This completes the proof.

			\qed

		\item Let $\alpha$ be any complex number. Let $\Lambda$ be the integer lattice.

			If $\inf_{z \in \mathbb{C} \setminus \Lambda} |\wp(z)-\alpha | = \varepsilon > 0$,
			then $1/(\wp(z)-\alpha)$ is bounded by $1/\varepsilon$ on $z \in \mathbb{C} \setminus \Lambda$.
			For $z \in \Lambda$, $1/(\wp(z)-\alpha) = 0$ so by the Riemann removable singularity theorem, $1/(\wp(z) -\alpha)$ is entire but bounded.
			So it must be constant which is contradiction.
			Thus, infimum of $|\wp(z) -\alpha|$ over $\mathbb{C} \setminus \Lambda$ equals to $0$ for any complex number $\alpha$.

			Let $\left\{ z_n \right\}_{n=1}^{\infty}$ be a sequence such that $|\wp(z_n) -\alpha | \rightarrow 0$ as $n\rightarrow \infty$.
			Since $\wp(z)$ is doubly periodic, by translating each $z_n$ appropriately, we can regard $\left\{ z_n \right\}$ as a sequence contained in the (closed) unit square whose vertices are $(0, 0), (0, 1), (1, 0), (1, 1)$.

			Then $\left\{ z_n \right\}$ is contained in compact set, so it has convergent subsequence $\left\{ z_{n_k} \right\}$.
			But $z_{n_k}$ cannot be converges to the vertices described above. Because $\wp(z)-\alpha = \infty$ at those vertices.
			So $z_{n_k}$ must converges to another points of the unit square described above.
			Then, by continuity, $|\wp(z) -\alpha| = 0$ for some $z \in$ [the unit square except the vertices].

			It leads surjectiveness of $\wp$.

			\qed

	\end{enumerate}
	
\end{problem}

\begin{problem}[10.5] \hfill
	\begin{enumerate}[label = (\alph*)]
		\item Fix $t$ and consider the following equation of $z$:
			\[
				\wp (z) = \gamma(t)
			\]
			This equation always has a solution since range of $\wp$ is $\mathbb{C}$.
			Let $\alpha$ be a solution of the above equation.
			Then $\wp '(\alpha) \ne 0$.
			So, (holomorphic) inverse function theorem can be applied.
			From $z_0$, we can analytically continue this function to $\gamma(1)$ along $\gamma$(detail: same as problem 1).
			Then $\Gamma: t \mapsto \wp^{-1}(\gamma(t))$ is what we want.

			Uniqueness directly follows from construction. Since $\wp(z) = \gamma(t)$ has simple solution thus $\wp$ is locally invertible.

		\item By definition of line integral,
			\[
				\begin{split}
					\int_\gamma \frac{dw}{\sqrt{4w^3 -C_1 w + C_2}}
					& = \int_0^1 \frac{\gamma'(t)}{\sqrt{4\gamma(t)^3 -C_1 \gamma(t) + C_2}}dt \\
					& = \int_0^1 \frac{\wp'(\Gamma(t)) \Gamma'(t)}{\wp'(\Gamma(t))}dt \\
					& = \int_0^1 \Gamma'(t) dt \\
					& = \Gamma(1) - \Gamma(0).
				\end{split}
			\]
	\end{enumerate}
	
	\qed
\end{problem}
