\begin{problem}[9.1] \hfill

	First, by considering the Maclaurine series of $\cos z$, $\cos \sqrt{z}$ is an entire function.
	Now, note that $\cos z = \frac{e^{iz} + e^{-iz}}{2}$, so modulus of $\cos \sqrt{z}$ is bounded by $e^{|z|/2}$.
	Therefore $\lambda(\cos \sqrt{z}) \leq 1/2$. Since genus is nonnegative integer bounded by order, the genus of $\cos \sqrt{z}$ must be 0.

	Now consider $f(z) = \sin^2 z$. Its zero set is $\{ k\pi \}$ where $k$ is an integer.
	Note that the smallest nonnegative integer $p$ satisfying $\sum_{k \ne 0} |k\pi|^{-p-1}$ is 1.
	Therefore the rank of $f$ is 1.
	Since $f(z) = \frac{e^{2iz} + e^{-2iz} -2}{-4}$, its modulus is bounded by $e^{2|z|}$.
	Thus $\lambda(f) \leq 1$.
	But we know the relation : $1= \text{rank} \leq \text{genus} \leq \text{order} \leq 1$.
	Therefore the genus of $\sin^2 z$ is one.

	Now consider $g(z) = \sin z^2$.
	The zero set of $g$ is $\{ \sqrt{k \pi} \}$ where $k$ is an integer.
	Note that the smallest nonnegative integer $p$ satisfying $\sum 2 |\sqrt{k \pi} |^{-p -1}$ is 2.
	Therefore the rank of $g$ is 2.
	Since $g(z) = \frac{e^{iz^2} - e^{-iz^2}}{2i}$, its modulus is bounded by $e^{|z|^2}$.
	Thus $\lambda(g) \leq 2$.
	So, $2 = \text{rank} \leq \text{genus} \leq \text{order} \leq 2$.
	
	\qed

\end{problem}

\begin{problem}[9.2] \hfill

	It is well known fact that $\left\{ e^{in\sqrt{2} \pi} : n \in \mathbb{N} \right\}$ is dense in $S^1$. 
	Let $a_n = \frac{2^n -1}{2^n} e^{in \sqrt{2} \pi}$.
	Then every point on $S^1$ is accumulation point of $\left\{ a_n \right\}_{n=1}^{\infty}$.
	Note that $\sum 1-|a_n| = \sum 2^{-n} < \infty$.
	Therefore the corresponding Blaschke product $B(z) = \Pi _n -\frac{\bar{a_n}}{|a_n|}B_{a_n}(z)$ is holomorphic on the unit disc $D$ and vanishes on $\left\{ a_n \right\}_{n=1}^\infty$ exactly.
	But, if $w \in \partial D$, then $w$ is accumulation point of the zero set of $B$.
	Thus if $w$ is regular, then extension of $B$ on small neighborhood of $w$ is identically zero, which is contradiction.
	So $B$ is the desired one.

	\qed
	
\end{problem}

Let $f$ be an entire function.
Let $M(r) = \sup_{|z| = r}|f(z)|$.
Before \#3 and \#4, we need the followings:
\[
	\limsup_{r \rightarrow \infty} \frac{\log \log M(r)}{\log r} = \lambda
\]
\[
	\limsup_{n\rightarrow \infty} \frac{n \log n}{-\log |a_n|} = \lambda
\]
where $a_n$ is the $n$-th Maclaurine coefficient of $f$. 

For the first formula, let $\rho < a = \limsup \frac{\log \log M(r)}{\log r}$.
Then there is $r_n \uparrow \infty$ such that $\rho < \frac{\log \log M(r_n)}{\log r_n}$.
Then $M(r_n) > \exp(r_n^{\rho})$ which says $\lambda \geq \rho$.
Since $\rho$ is arbitrary, we can deduce that $\lambda \geq a$.

For the other direction, let $\rho < \lambda$.
Then there is increasing sequnce $r_n \uparrow \infty$ such that $M(r_n) > \exp(r_n ^\rho)$.
Thus $\log \log M(r_n) / \log r_n \geq \rho$ which leads $a \geq \rho$.
Since $\rho \leq \lambda$ is arbitrary, $a \geq \lambda$.

For the second formula, let $\mu = \limsup_n \frac{n\log n}{-\log |a_n|}$.
If $\mu = \infty$, then $\lambda \leq \mu$ directly. So assume $\mu < \infty$ and $\varepsilon>0$.
Then $0 \leq \frac{n\log n }{-\log |a_n|} \leq \mu + \varepsilon$ for $n \geq N$.
By simple calculation, $|a_n| \leq n^{-n/(\mu + \varepsilon)}$.
Thus $M(r) \leq \sum |a_n| r^n \leq \sum n^{-n/(\mu+\varepsilon)}r^n = \sum_{n <(2r)^{\mu + \varepsilon}} ( ) + \sum_{n \geq (2r)^{\mu+\varepsilon}}( ) = S_1 + S_2$. \\
\[
	\begin{split}
		S_1 & \leq r^{(2r)^{\mu+\varepsilon}} \sum_n n^{-n/(\mu+\varepsilon)} \\
		& = O(r^{(2r)^{\mu+\varepsilon}} = O(\exp( (2r)^{\mu+\varepsilon} \log r) ) \\
			& = O(\exp(r^{\mu + 2\varepsilon}))
	\end{split}
\]

And $n^{-1/(\mu+\varepsilon)}r \leq 1/2$ yields $S_2 \leq 1$.
Thus $M(r) = O(\exp(r^{\mu+2\varepsilon}))$, which implies $\lambda \leq \mu + 2\varepsilon$.
By letting $\varepsilon \downarrow 0$, we get $\lambda \leq \mu$.

For the other direction, let $0 < \tau < \mu$.
Then $\tau \leq \frac{n \log n}{-\log |a_n|}$ for infinitely many $n$ which goes to $\infty$.
For those $n$, $\log |a_n| \geq \frac{-n\log n}{\tau}$.
By cauchy's thm, we know that $|a_n| \leq M(r) r^{-n}$.
So, 
\[
\begin{split}
	\log M(r)
& \geq \log|a_n| + n\log r \\
& \geq n\left( \log r - \frac{\log n}{\tau} \right)
\end{split}
\]
By taking $r_n = (en)^{1/\tau}$, $\log M(r_n) \geq n/\tau = r_n^\tau / (e\tau)$.
So 
\[
	\frac{\log \log M(r_n)}{\log r_n} \geq \frac{\tau \log r_n - \log e\tau}{\log r_n}
\]
thus $\limsup \geq \tau$.
Since $\tau$ is arbitrary, we get $\lambda \geq \mu$ by the first formula.

\qed

\begin{problem}[9.3] \hfill

	If $\sum a_n z^n$ is an entire function, then its order is determined by $\limsup_{n\rightarrow \infty} \frac{n \log n}{-\log |a_n|}$.
	\begin{enumerate}[label = (\alph*)]
		\item First represent $f$ as the Maclaurine series.
			Let $a_n$ be its $n$-th coefficient. 
			But $\limsup_n \frac{n\log n}{-\log n -\log |a_n|} = \limsup_n \frac{n \log n}{- \log |a_n|}$.
			So the order of $f$ and $f'$ are same.
		\item Note that $\log E_n(z) = z^{n+1}/(n+1) + z^{n+2}/(n+2) + \cdots$ by power series.
			Also, $\log|z| \leq |\log z| = |\log |z| + i arg(z)|$.
			So $\log |E_n(z)| \leq |z|^{n+1}/(1-|z|)$ for $|z| < 1$.

			By definition of $E_n$, it is also clear that $\log |E_n| \leq \log |E_{n-1}| + |z|^n$.
			Now we claim that $\log |E_n| \leq (2n+1) |z|^{n+1}$.
			This can be done by the following:
			\[
				\begin{split}
					\log |E_n|
					& \leq |z|\log|E_n| + |z|^{n+1} \\
					& \leq |z|(\log|E_{n-1}| + |z|^n) + |z|^{n+1}\\
					& \leq |z|(2n |z|^n) + |z|^{n+1}
				\end{split}
			\]
			for $|z| < 1$ and induction.
			The case when $|z| \geq 1$ can be done by using the part of above.

			Now put $n = \mu = $genus.
			Let $P$ be the canonical product of given entire function with rate $\mu$.
			Then $\log |P| \leq (2\mu + 1) |z|^{\mu + 1} \sum_n |a_n|^{-\mu -1}$.
			Since $f = c z^m e^g P$ where the degree of $g$ is less or equal to $\mu$,
			the order of $f$ is thus determined by $P$.
			The above inequality implies $\lambda(f) \leq \mu+1$.

		\item Let $a_n$ be a sequence of zeros of $f$. Since we know that the order of $f$ and $f'$ are same, $\lambda(f) \leq 1$.
		Thus $\sum_n |a_n|^{-1 -1} < \infty$. But $\sum_n (\sqrt{n})^{-1 -1} \leq \sum_n |a_n|^{-1 -1} < \infty$ which is contradiction.
			Therefore $f$ must be constant, so $f(z) = 0$ for every $z$.

			\qed
	\end{enumerate}
\end{problem}

\begin{problem}[9.4] \hfill

	Let $a_n$ be $n$-th coefficient of $g$. Then $\limsup_{n\rightarrow \infty} |a_n|^{1/n} = 0$ so the radius of convergence is $\infty$, thus $g$ is an entire function.

	By Stirling's formula, $\log (n!) = n \log n -n + O(\log n)$. Therefore $\frac{n\log n}{\log (n!)} \rightarrow 1 $ as $n \rightarrow \infty$.	

	$\frac{n\log n}{-\log a_n} = \frac{n \log n }{\alpha \log(n!)} \rightarrow 1/\alpha$ as $n\rightarrow \infty$. Therefore the order of $g$ is $1/\alpha$.

	\qed
\end{problem}

\begin{problem}[9.5] \hfill

	By considering the Maclaurine series of $\sin z$, $\sin \sqrt{z} / \sqrt{z}$ is holomorphic by the Riemann removable singularity theorem.
	And by simple calculation, its order is bigger than $0$ and smaller or equal to $1/2$.

	Now, consider $f(z) = \sin z / z$. Since the order of $f$ is finite and $f$ is entire, it can omit at most one complex number.
	If $f$ omit the value $c$,then $f(z) -c$ is nonvanishing, so $f(z) - c = \exp (g(z))$.
	But the degree of $g$ must be $0$ or $1$ since the order of $f$ is less or equal to $1$.
	If the degree of $g$ is zero, then $f(z) - c$ is constant which is contradiction.
	So we can say that $f(z) - c = \exp(az+b)$.
	But, as $|z| \rightarrow \infty$, $\left | \frac{f(z) - c}{\exp(az+b)} \right | \rightarrow 0$ which is contradiction because it must be equal to $1$.
	Therefore, we can conclude that $f(z)$ assumes every complex value.

	Let $c\in \mathbb{C}$ be given. Then the solution of $f(z) = c$ exists, say $\alpha$. Then $\alpha^2$ is a solution of $f(\sqrt z) = c$.
	Therefore $c$ is in the image of $f(\sqrt{z})$, which is entire of nonintegral finite order.
	Thus there are infinitely many solutions of $f(\sqrt{z}) = c$, say $w_1, w_2, \cdots$.
	Then $\sqrt{w_1}, \sqrt{w_2}, \cdots$ are the infinite solutions of $f(z) = c$.

	\qed
	
\end{problem}
