\begin{problem}[4.7.2] \hfill

	Let $w_N = \sup \left\{ |Y_n - Y_m | : m, n \leq N \right\}$.
	Then clearly $|Y_n - Y_{-\infty} | \leq w_N$ for $n \leq N$.
	Since $Y_n \rightarrow Y_{-\infty}$ a.s. as $n \rightarrow -\infty$,
	$w_N \rightarrow 0$ as $N \rightarrow -\infty$ almost surely.
	Note that $w_N \leq 2Z \in L_1$.
	Thus $w_N \in L_1$.
	So, $\mathbb{E}(w_N | \mathcal{F}_{-\infty}) \rightarrow 0$ as $N\rightarrow -\infty$.

	Let 
	\[
		\begin{split}
			& \left | E(Y_n |\mathcal{F}_n) - E(Y_{-\infty} | \mathcal{F}_{-\infty} ) \right | \\
			& \leq |E(Y_n |\mathcal{F}_n) - E(Y_{-\infty} | \mathcal{F}_n) |
			+ |E(Y_{-\infty} | \mathcal{F}_n ) - E(Y_{-\infty} | \mathcal{F}_{-\infty})| \\
			& = S_1 + S_2
		\end{split}
	\]

	Now, consider the following:
	\[
\begin{split}
	\limsup_{n\rightarrow -\infty} \left | E(Y_n | \mathcal{F}_n) - E(Y_{-\infty} | \mathcal{F}_n) \right |
	& \leq \limsup E(\left | Y_n - Y_{-\infty} \right | | \mathcal{F}_n ) \\
	& \leq \lim_{n \rightarrow -\infty} E(w_N | \mathcal{F}_n) \\
	& = E(w_N |\mathcal{F}_{-\infty})
\end{split}
	\]
	So, by $N \rightarrow -\infty$, $\limsup_{n\rightarrow -\infty} S_1 = 0$.
	$\limsup_{n\rightarrow -\infty} S_2 = 0$ because $E(Y_{-\infty} | \mathcal{F}_n)$ is a backward martingale so it converges to $E(Y_{-\infty}| \mathcal{F}_{-\infty})$ a.s. and in $L_1$.

	\qed
\end{problem}

\begin{problem}[4.8.1]\hfill

	Let $K_n = 1_{L < n \leq M}$. Then $(K\cdot X)_n = X_{n\wedge M} - X_{n\wedge L}$ is a submartingale.
	Thus $EX_{n\wedge L} \leq EX_{n\wedge M}$.
	Similarly, $EX_{n\wedge L}^+ \leq EX_{n\wedge M}^+$.
	Since $X_{n \wedge M}$ is uniformly integrable submartingale, $EX_{n\wedge M} \rightarrow EX_{M}$ as $n\rightarrow \infty$.
	If we can show that $X_{n\wedge L}$ is also a uniformly integrable submartingale, then $EX_L \leq EX_M$.

	Note that $\sup_n EY_{n\wedge L}^+ \leq \sup_n EY_{n\wedge M}^+ \leq \sup_n E|Y_{n\wedge M}| < \infty$ by uniform integrability of $Y_{n\wedge M}$.
	Thus, by martingale convergence theorem, we get $Y_{n\wedge L} \rightarrow Y_L$ almost surely and $Y_L \in L_1$.

	Now it remains to show that $Y_n 1_{n < L}$ is uniformly integrable.
	\[
		E(|Y_{n\wedge L}|;|Y_{n\wedge L}|>K, n<L) = E\left( |Y_{n\wedge M}|; |Y_{n\wedge M}|>K, n<L\leq M \right)
	\]
	And $\sup$ of the last term goes to $0$ as $K \rightarrow \infty$ by uniform integrability of $Y_{n\wedge M}$.
	Therefore, by theorem 4.8.2, $Y_{n\wedge L}$ is uniformly integrable, hence $EY_L \leq EY_M$.


	
\end{problem}<++>
\begin{problem}[4.8.4] \hfill

	Let $M_n = S_n ^2 - n\sigma^2$. Then $M_n$ is a quadratic martingale.
	Since $n\wedge T$ is bounded stopping time, we have $EM_{n\wedge T} = EM_0 = 0$.
	Thus $ES_{n\wedge T}^2 = \sigma^2 E(n\wedge T)$.
	As $n\rightarrow \infty$, $E(n\wedge T) \rightarrow ET$ by MCT.

	Now consider $E|S_T - S_{n \wedge T}|^2 = E\left( \sum_{m = n+1}^\infty 1_{(m \leq T)} \xi_m \right)^2$.
	Note that 
	\[
		E (1_{m \leq T}) (1_{m+k \leq T}) \xi_m \xi_{m+k} = (E\xi_{m+k})E1_{m\leq T}1_{m+k\leq T}\xi_m = 0
	\]
	Thus $E|S_T - S_{n\wedge T}|^2 = \sum_{m=n+1}^\infty E1_{(m\leq T)}\xi_m^2 = \sigma^2 \sum_{m=n+1}^\infty P(m\leq T)$.
	But, $\sum_{m=1}^\infty P(m\leq T) = ET <\infty$.
	So we can say $E|S_T - S_{n \wedge T}|^2 \rightarrow 0$ as $n\rightarrow 0$.
	
	Therefore $S_{n\wedge T} \rightarrow S_T$ in $L_2$, so $S_{n\wedge T}^2 \rightarrow S_T^2$ in $L_1$, which leads the conclusion.

	\qed

\end{problem}

\begin{problem}[4.8.7] \hfill

	Note that $ET = a^2$ by theorem 4.8.7.
	Claim : $(b, c) = (3, 2)$.
	\[
		\begin{split}
			& E(Y_{n+1}|\mathcal{F}_n) \\
			& = 1+6S_n^2 +S_n^4 -6(n+1)(1+S_n^2) + 3(n+1)^2 +2(n+1) \\
			& = S_n^4 -6n S_n^2 + 3n^2 +2n
		\end{split}
	\]
	Since $n\wedge T$ is bounded stopping time, we can get $EY_0 = EY_{n\wedge T}$.
	Thus $3E(n\wedge T)^2 = 6E\left[ (n\wedge T)S_{n\wedge T}^2 \right] - ES_{n\wedge T}^4 - 2E_{n\wedge T}$.
	
	But, by MCT, $E(n\wedge T)^2 \rightarrow ET^2$ and $E(n\wedge T) \rightarrow ET$.
	And $S_{n\wedge T}$ is bounded, so BCT implies $ES_{n\wedge T}^m \rightarrow a^m$.
	Thus, $3ET^2 = 6a^4 - a^4 - 2ET = 5a^4 -2a^2$.

	Note that $(n\wedge T)S_{n\wedge T}^2 \leq a^2T \in L_1$. Thus $E(n\wedge T)S_{n\wedge T}^2 \rightarrow a^4$ by DCT.

	\qed	
\end{problem}

\begin{problem}[4.8.9] \hfill
	
	Since $n \wedge T$ is a bounded stopping time, $EX_0 = EX_{n \wedge T} = EX_n = 1$ which is sames as follows:
	\[
		1 = EX_n 1_{(n < T)} + EX_T 1_{(n \geq T)} = EX_{n} 1_{(n < T)} + \exp(\theta_0 a)P(n \geq T)
	\]
	But, by MCT, the last term of the above $\rightarrow \exp(\theta_0 a) P(T < \infty)$.

	Since $E\xi_i > 0$, $ES_n \rightarrow \infty$ as $n\rightarrow \infty$.
	Then, since $x \mapsto \exp(x)$ is a convex function,
	\[
		E \exp(\theta_0 S_n) \leq \exp(\theta_0 ES_n).
	\]
	Thus $EX_n 1_{(n< T)} \leq EX_n \leq \exp(\theta_0 ES_n) \rightarrow 0$ as $n\rightarrow \infty$ due to $\theta_0 <0$.

	Therefore $1 = \lim_n EX_n 1_{(n < T)} +\exp(\theta_0 a)P(T<\infty) = \exp(\theta_0 a) P(T < \infty)$, which says the conclusion.

	\qed

\end{problem}
