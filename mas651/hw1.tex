\begin{problem}[5.1.1] \hfill

	Let $(S, \mathcal{S})$ be a state space of $X_n$ where $S = \left\{ 1, 2, \cdots N \right\}$ and $\mathcal{S} = 2^S$.
	Note that $N$ is an absorbing state.
	And $X_1 = 1$ with probability $1$.
	For fixed $k$ such that $1 \leq k < N$, $k \leq n$,
	\[
		P(X_{n+1} = k+1 \lvert X_n = k) = \frac{N-k}{N}
	\]
	and
	\[
		P(X_{n+1} = k \lvert X_n = k) = \frac{k}{N}.
	\]
	If $k > n$, then the above are all $0$.
	So it is a temporally inhomogeneous.
	The Markov property is trivial since the very next state only depends on the current state.

	\qed

\end{problem}

\begin{problem}[5.1.2] \hfill

	\[
		P(X_4 = 2 \lvert X_3 = 1, X_2 = 1, X_1 = 1, X_0 = 0) = (1/16)/(1/4) = 1/4
	\]
	but
	\[
		P(X_4 = 2 \lvert X_3 = 1, X_2 = 0, X_1 = 0, X_0 = 0) = (1/16)/(1/8) = 1/2.
	\]
	Thus $X_n$ is not a Markov chain.

	\qed
\end{problem}

\begin{problem}[5.1.5] \hfill

	\[
		P(X_{n+1} = k+1 \lvert X_n = k) = \frac{m-k}{m}\frac{b-k}{m}
	\]
	because we must choose a white ball in the left urn and a black ball in the right urn.
	\[
		P(X_{n+1} = k \lvert X_n = k) = \frac{k}{m}\frac{b-k}{m} + \frac{m-k}{m}\frac{m+k-b}{m}
	\]
	since there are two cases, choosing both black or both white.
	\[
		P(X_{n+1} = k-1 \lvert X_n = k) = \frac{k}{m}\frac{m+k-b}{m}
	\]
	since we must choose a black ball in the left urn and a white ball in the right urn.
	Note that the sum of the above is 1, so there is no other transition probability.

	\qed
	
\end{problem}

\begin{problem}[5.1.6] \hfill
	
	\[
		P(S_{n+1} = k+1 \lvert S_n = k) = \frac{P(X_{n+1} = 1, S_n = k)}{P(S_n = k)}
	\]
	where the denominator is
	\[
		\int_{\theta \in (0, 1)}P(S_n = k \lvert \theta) dP = {n \choose x} \frac{x!y!}{(n+1)!} = \frac{1}{n+1}
	\]
	for $x = $the number of $i$ such that $U_i \leq \theta$ and $y = n-x$.
	Note that $x = (n+k)/2$ and $y = (n-k)/2$ since $x+y = n$ and $x-y = k$.
	The numerator is
	\[
		\int_{\theta \in (0, 1)}P(X_{n+1} = 1, S_n = k \lvert \theta)dP = {n \choose x}\frac{(x+1)!y!}{(n+2)!}
	\]

	These are because $P(S_n = k \lvert \theta) = \theta^x (1-\theta)^y {n \choose x}$ and $P(X_{n+1} = 1, S_n = k \lvert \theta) = P(X_{n+1} = 1 \lvert \theta)P(S_n = k \lvert \theta) = {n \choose x} \theta^{x+1}(1-\theta)^y$ and using the kernel of beta distribution.

	Thus, the probability what we want is $(n+k+2)/(2n+4)$ which depends on $n$.
	So $X_n$ is temporally inhomogeneous.
	\[
		P(S_{n+1} = k+1 \lvert S_1 = t_1, \cdots, S_n = k) = P(X_{n+1} = 1 \lvert X_1 = t_1, \cdots, X_n = t_n)
	\]
	where $\sum_{i=1}^n t_i = k$.
	We can show the above is equal to $P(S_{n+1} = k+1 \lvert S_n = k) = (n+k+2)/(2n+4)$ similarly, by omitting the ${n \choose x}$ term of both denominator and numerator.

	\qed
\end{problem}
