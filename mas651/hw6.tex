\begin{problem}[6.1.1] \hfill
	
	Let $\varphi$ be a measure preserving map of $\Omega$ into $\Omega$.
	First, $\varphi^{-1}\Omega = \Omega$ since codomain contains the range.
	Second, if $A$ is an invariant set, then $\varphi^{-1}A^c = \left( \varphi^{-1}A \right)^c = A^c$, so the complement of $A$ is invariant.
	Third, if $A_i$ is an increasing sequence of invariant set, then $\varphi^{-1}A = \cup_i \varphi^{-1}A_i = \cup_i A_i = A$, so $A = \cup_i A_i$ is also invariant.
	Therefore $\mathcal{I}$ is a sigma field.

	We say two sets $A, B$ are equal a.s. if their corresponding indicator functions are equal a.s.

	Let $B$ be a Borel set.
	Assume that $X \circ \varphi = X$ a.s.
	Then $\varphi^{-1}X^{-1}B = X^{-1}B$ a.s. so $X^{-1}B$ is invariant,
	which says $X$ is $\mathcal{I}$ measurable.
	Let's consider the converse.
	If $X$ is an indicator function, then the result trivially holds.
	So we can extend the result to where $X$ is an simplie function, measurable with respect to $\mathcal{I}$.
	If $X$ is nonnegative function of $\mathcal{I}$, then $s_n \uparrow X$.
	Since $s_n \circ \varphi = s_n$ a.s, $X\circ \varphi = \lim_n s_n \circ \varphi = \lim_n s_n = X$ a.s.
	If $X \in \mathcal{I}$ any r.v, then by decomposing it to $X = X^+ - X^-$, we can conclude the result.

	\qed
\end{problem}

\begin{problem}[6.1.2] \hfill

	\begin{enumerate}
		\item 
			\[
				\begin{split}
					\omega \in \varphi^{-1}(B)
					&\Rightarrow \varphi(\omega) \in B
					\Rightarrow \varphi(\omega) \in \varphi^{-n}(A)
					\Rightarrow \varphi^{n+1}(\omega) \in A \\
					&\Rightarrow \omega \in \varphi^{-n-1}(A)
					\Rightarrow \omega \in B
				\end{split}
			\]
			for some $n \ge 0$.
			Therefore $\varphi^{-1}(B) \subset B$.

		\item
			\[
\begin{split}
	\omega \in \varphi^{-1}(C) &\Rightarrow \varphi(\omega) \in \varphi^{-n}(B)
	\Rightarrow \varphi^{n+1}(\omega) \in B \\
	&\Rightarrow \omega \in \varphi^{-n-1}(B) \Rightarrow \omega \in \varphi^{-n}(B)
\Rightarrow \omega \in C
\end{split}
	\]
	for all $n \ge 0$.
	Therefore $\varphi^{-1}(C) \subset C$.

	\[
		\begin{split}
			C &= B \cap \varphi^{-1}(B) \cap \cdots \\
			& \subset \varphi^{-1}(B) \cap \varphi^{-2}(B) \cap \cdots .
		\end{split}
	\]
	Because $H\cap G \subset H$.
	Therefore $C = \varphi^{-1}(C)$.

\item Let $B, C$ be same as above.
	Assume that $A$ is almost invariant.

	By elementary set theory, we have the followings:
	\[
		\begin{split}
			A \Delta C & \subset (A \Delta B) \cup (B \Delta C) \\
			A \Delta (B\cup  C) & \subset (A \Delta B) \cup (A \Delta C)\\
			A \Delta (B\cap  C) & \subset (A \Delta B) \cup (A \Delta C).
		\end{split}
	\]
	Since $\varphi^{-n}(A) = \varphi^{-n-1}(A)$, by the first relation, $A = \varphi^{-n}(A)$ a.s.
	Let $B_n = \bigcup_{k = 0}^n \varphi^{-k}(A)$.
	Then by the second relation, $A = B_n$ a.s.
	But $B_n \uparrow B$, so
	\[
		P(A\Delta B) \le \sum P(A\Delta B_n) = 0
	\]
	by the second relation.
	So $A = B$ a.s.
	Similarly, by using the third relation, we can see that $B = C$ a.s.
	Thus $A = B = C$ a.s, by the first relation, and $C$ is strictly invariant by (1,2).
	

	Conversely, assume that $P(A \Delta D) = 0$ for some strictly invariant $D$.
	Since $\varphi$ is measure preserving, $P(\varphi^{-1}(A \Delta D)) = P(A \Delta D) = 0$.
	So $\varphi^{-1}(A) = \varphi^{-1}(D)$ a.s.
	But $A = C = \varphi^{-1}(C)$ a.s.
	Thus $A = \varphi^{-1}(A)$a.s, equivalent to almost invariance of $A$.

	\end{enumerate}

	\qed
\end{problem}

\begin{problem}[6.1.4] \hfill
	
	Let $m$ be any integer, $n$ be any nonnegative integer.
	Define
	\[
		\mu_{m, \cdots, m+n}\left( A_0, \cdots, A_n \right)
		= P(X_0 \in A_0, \cdots, X_n \in A_n).
	\]
	Then $\mu_{m, \cdots, m+n}$ is consistent, so the Kolmogorov extension thm applies.
	Let $Y_n$ be a coordinate map.
	Then any length $n+1$ distribution of consecutive sequence of $Y$ has same distribution with $X_0, \cdots, X_n$.
	It means $Y_n$ is a two sided stationary process, and $X_n$ is embeded in $Y_n$

	\qed.
\end{problem}

\begin{problem}[6.2.1] \hfill

	Assume $X \in L^p$.
	Let $A_n(X_M') = \sum_{m=0}^{n-1}X_M' \circ \varphi^m /n$.
	Since $A_n(X_M') \rightarrow E(X_M' \lvert \mathcal{I})$ a.s. and
	$|A_n(X_M') - E(X_M'\lvert \mathcal{I}) |^p \le (2M)^p \in L^1$,
	DCT implies $L^p$ convergence of $A_n(X_M') \rightarrow E(X_M'\lvert \mathcal{I})$.

	Now consider $\| A_n(X_M'') - E(X_M'' \lvert \mathcal{I}) \|_p \le \| A_n(X_M'')\|_p + \|X_M''\|_p$. \\
	But $\| A_n(X_M'')\|_p \le \sum_{m=0}^{n-1} \| X_M'' \circ \varphi^m \|_p /n = \|X_M''\|_p$.
	Since $|X_M''|^p \le |X|^p \in L^1$ and $X_M'' \rightarrow 0$ a.s,
	DCT implies $L^p$ convergence of $A_n(X_M'') \rightarrow E(X_M'' \lvert \mathcal{I})$.

	\qed
\end{problem}

\begin{problem}[6.2.2] \hfill

	\begin{enumerate}
		\item Fix $M>0$.
			Let $h_M = \sup_{m \ge M} |g_m - g|$.
			By our assumption, $h_M \in L^1$ and $h_M \downarrow 0$ a.s. as $M \uparrow \infty$.
			Since $g\in L^1$,
			\[
				{1 \over n}\sum_{m=0}^{n-1}g_m\circ\varphi^m \rightarrow E(g \lvert \mathcal{I})\text{ a.s}
			\]
			So, it is sufficient to show that
			\[
				\frac{1}{n}\sum_{m=0}^{n-1}(g_m-g)\circ\varphi^m \rightarrow 0 \text{ a.s.}
			\]
			Consider
			\[
				\frac{1}{n}\sum_{m=0}^{n-1}|g_m - g|\circ\varphi^m
				\le \frac{1}{n}\sum_{m=0}^{M-1}|g_m - g|\circ\varphi^m
				+ \frac{1}{n}\sum_{m=M}^{n-1}h_M \circ \varphi^m.
			\]
			By taking $\limsup_{n\rightarrow \infty}$ on both sides,
			\[
				\limsup_{n\rightarrow \infty}\frac{1}{n}\sum_{m=0}^{n-1}|g_m-g|\circ\varphi^m
				\le E(h_M \lvert \mathcal{I}).
			\]
			By theorem 4.1.9(Monotone convergence theorem for conditional expectation),
			as $M \uparrow \infty$, the last term goes to $0$ a.s.

		\item Since $g \in L^1$,
			\[
				\frac{1}{n}\sum_{m=0}^{n-1}g\circ\varphi^m \rightarrow  E(g\lvert \mathcal{I})
			\]
			a.s. and in $L^1$ by the Ergodic theorem.
			Now, it is sufficient to show that
			\[
				\frac{1}{n}\sum_{m=0}^{n-1}(g_m - g)\circ \varphi^m \rightarrow 0
			\]
			in $L^1$ sense.
			Fix $\varepsilon>0$.
			Since $\| g_n -g \|_1 \rightarrow 0$, we can choose $N$ such that
			$\| g_n - g \|_1 < \varepsilon$ whenever $n \ge N$.
			Then,
			\[
				\left \| \frac{1}{n}\sum_{m=0}^{n-1}(g_m - g)\circ \varphi^m \right \|_1 \le 
				\frac{1}{n}\sum_{m=0}^N \| g_m - g \|_1 + \varepsilon.
			\]
			If $n$ is sufficiently large, then the above is bounded by $\varepsilon$.
			Since $\varepsilon$ is arbitrary, the above goes to $0$ in $L^1$ sense.
	\end{enumerate}

	\qed
\end{problem}

\begin{problem}[6.2.3] \hfill

	Note that
	\[
		D_k - \alpha > 0
		\Leftrightarrow \sup_{i\le k}\frac{S_i -i\alpha}{i} > 0
		\Leftrightarrow \sup_{i\le k}\frac{\sum_{j=0}^{i-1}(X_j - \alpha)}{i} >0
		\Leftrightarrow \sup_{i\le k}\sum_{j=0}^{i-1}(X_j -\alpha) >0.
	\]
	Let the last condition be $M_k > 0$.
	Then the above says $D_k - \alpha > 0$ is equivalent to $M_k >0$.
	Therefore by lemma 6.2.2,
	\[
		0 \le E\left[ (X-\alpha)1_{\left( M_k >0 \right)} \right].
	\]
	Thus,
	\[
		\alpha P(D_k > \alpha) \le EX1_{(M_k>0)} \le E|X|.
	\]

	\qed
\end{problem}
