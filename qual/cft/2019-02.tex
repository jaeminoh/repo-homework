\section*{2019.02}
\begin{problem}[Casorati-Weierstrass]

	If the image of $f$ is not dense in $\mathbb{C}$, then there are $\varepsilon >0$ and $w \in \mathbb{C}$ such that
	\[
		\lvert f(z) - w \lvert > \varepsilon
	\]
	for all $z \in D'(z_0, r)$.
	Now consider $g(z) = 1/(f(z) - w)$.
	Then the modulus of $g$ is bounded by $1/\varepsilon$.
	So the Riemann removable singularity theorem implies that $g \in H(D(z_0, r))$.

	If $g(z_0) = 0$, then $f$ has a pole at $z_0$, which is contradiction.
	If $g(z_0) \ne 0$, then $f$ must be bounded near $z_0$, which contradicts to the essential singularity.

	\qed
\end{problem}

\begin{problem}
	
	Observe that the given polynomial is a partial sum of $\exp(z)$.
	Since the radius of convergence of the power series of $\exp(z)$ is $\infty$,
	the given polynomial converges locally uniformly.

	Note that $ \lvert \exp(z) \lvert \geq \exp(-R)$ on $z \in \partial D(0, R)$.
	Thus, if we take $n$ so large that 
	\[
		\lvert P_n(z) - \exp(z) \lvert < \exp(-R)
	\]
	for all $z \in \partial D(0, R)$, then Rouche's theorem implies the result because $\exp(z)$ is nonvanishing.

	\qed
\end{problem}

\begin{problem}
	
	Since the modulus of $f$ is $1$ on the boundary of the unit disk, the modulus of $f$ is bounded by $1$ on the entire unit disk.
	Thus, by the maximum modulus principle, $f$ is a self mapping of the unit disk.
\end{problem}

\begin{problem}
	
	Fix $r>0$ and choose $N$ such that $\lvert a_n \lvert > 2r$ whenever $n \geq N$.
	Then
	\[
		\sum_{n\geq N} \left \lvert \frac{r}{a_n} \right \lvert ^n \leq \sum_{n\geq N} \left( \frac{1}{2} \right)^n < \infty.
	\]
	Thus, for each $r>0$,
	\[
		\sum_{n \in \mathbb{N}} \left \lvert \frac{r}{a_n} \right \lvert ^n < \infty.
	\]
	This implies
	\[
		\prod_{n \in \mathbb{N}} E_{n-1}\left( \frac{z}{a_n} \right)
	\]
	is an entire function.

	(explanation about the zeros are needed)

	\qed
\end{problem}

\begin{problem}
	
	If $f(0) = 0$, then the result follows trivially.
	So assume that $f(0) \ne 0$.
	Consider
	\[
		g(z) = \frac{f(z)}{\prod_{k=1}^{n(R)}B_{a_k/R}(z/R)}
	\]
	where $n(R)$ denotes the number of zeros of $f$ in $D(0, R)$.
	Then $g$ is nonvanishing.
	So $\log |g|$ is harmonic, and the mean value property implies
	\[
		\log|g(0)| = \frac{1}{2\pi}\int_0^{2\pi}\log|g(Re^{i\theta})| d\theta
	\]
	which is equivalent to
	\[
		\log|f(0)| - \sum_{k=1}^{n(R)}\log \left \lvert \frac{a_k}{R} \right \lvert = \frac{1}{2\pi} \int_{0}^{2\pi} \log |f(Re^{i\theta})| d\theta.
	\]
	Thus, the result follows immediately if we observe that $|a_k / R| \leq |r/ R| \leq 1$ and $n(R) \geq n(r) \geq n$.

	\qed
\end{problem}

\begin{problem}
	
	Let $K$ be a compact subset of the unit disk.
	Then we can find $0\leq r < 1$ such that $K \subset \overline{D}(0, r)$.
	Note that
	\[
		|f(z)| \leq \sum_{n\geq 1}|a_n||z|^n \leq \sum_{n\geq 1 }nr^n < \infty.
	\]
	Thus, $\mathcal{F}$ is locally uniformly bounded.
	Then second Montel's theorem implies the result.

	\qed
\end{problem}
